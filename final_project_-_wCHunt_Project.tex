\documentclass[]{article}
\usepackage{lmodern}
\usepackage{amssymb,amsmath}
\usepackage{ifxetex,ifluatex}
\usepackage{fixltx2e} % provides \textsubscript
\ifnum 0\ifxetex 1\fi\ifluatex 1\fi=0 % if pdftex
  \usepackage[T1]{fontenc}
  \usepackage[utf8]{inputenc}
\else % if luatex or xelatex
  \ifxetex
    \usepackage{mathspec}
  \else
    \usepackage{fontspec}
  \fi
  \defaultfontfeatures{Ligatures=TeX,Scale=MatchLowercase}
\fi
% use upquote if available, for straight quotes in verbatim environments
\IfFileExists{upquote.sty}{\usepackage{upquote}}{}
% use microtype if available
\IfFileExists{microtype.sty}{%
\usepackage{microtype}
\UseMicrotypeSet[protrusion]{basicmath} % disable protrusion for tt fonts
}{}
\usepackage[margin=1in]{geometry}
\usepackage{hyperref}
\hypersetup{unicode=true,
            pdftitle={DATA 609 - Final Project},
            pdfauthor={Daina Bouquin, Christophe Hunt, Christina Taylor},
            pdfborder={0 0 0},
            breaklinks=true}
\urlstyle{same}  % don't use monospace font for urls
\usepackage{graphicx,grffile}
\makeatletter
\def\maxwidth{\ifdim\Gin@nat@width>\linewidth\linewidth\else\Gin@nat@width\fi}
\def\maxheight{\ifdim\Gin@nat@height>\textheight\textheight\else\Gin@nat@height\fi}
\makeatother
% Scale images if necessary, so that they will not overflow the page
% margins by default, and it is still possible to overwrite the defaults
% using explicit options in \includegraphics[width, height, ...]{}
\setkeys{Gin}{width=\maxwidth,height=\maxheight,keepaspectratio}
\IfFileExists{parskip.sty}{%
\usepackage{parskip}
}{% else
\setlength{\parindent}{0pt}
\setlength{\parskip}{6pt plus 2pt minus 1pt}
}
\setlength{\emergencystretch}{3em}  % prevent overfull lines
\providecommand{\tightlist}{%
  \setlength{\itemsep}{0pt}\setlength{\parskip}{0pt}}
\setcounter{secnumdepth}{5}
% Redefines (sub)paragraphs to behave more like sections
\ifx\paragraph\undefined\else
\let\oldparagraph\paragraph
\renewcommand{\paragraph}[1]{\oldparagraph{#1}\mbox{}}
\fi
\ifx\subparagraph\undefined\else
\let\oldsubparagraph\subparagraph
\renewcommand{\subparagraph}[1]{\oldsubparagraph{#1}\mbox{}}
\fi

%%% Use protect on footnotes to avoid problems with footnotes in titles
\let\rmarkdownfootnote\footnote%
\def\footnote{\protect\rmarkdownfootnote}

%%% Change title format to be more compact
\usepackage{titling}

% Create subtitle command for use in maketitle
\newcommand{\subtitle}[1]{
  \posttitle{
    \begin{center}\large#1\end{center}
    }
}

\setlength{\droptitle}{-2em}
  \title{DATA 609 - Final Project}
  \pretitle{\vspace{\droptitle}\centering\huge}
  \posttitle{\par}
  \author{Daina Bouquin, Christophe Hunt, Christina Taylor}
  \preauthor{\centering\large\emph}
  \postauthor{\par}
  \predate{\centering\large\emph}
  \postdate{\par}
  \date{April 16, 2017}

\usepackage{relsize}
\usepackage{setspace}
\usepackage{amsmath,amsfonts,amsthm}
\usepackage[sfdefault]{roboto}
\usepackage[T1]{fontenc}
\usepackage{float}
\usepackage{multirow}
\usepackage{mathtools}
\usepackage{tikz}
\usetikzlibrary{shapes,arrows}

\begin{document}
\maketitle

{
\setcounter{tocdepth}{2}
\tableofcontents
}
\subsection{Textbook Part III:}\label{textbook-part-iii}

\subsubsection{The problem}\label{the-problem}

The digestive processes of sheep can highlight the nutrionaly value in
varied feeding schedules or varied food prepation. This is esspecially
important when raising sheep for commercial purposes.

\subsubsection{The digestive process}\label{the-digestive-process}

Sheep are a cud-chewing animal which means that unchewed food goes
through a series of storage stomachs called the rumen and the reticulum.
The process is illustrated below:

\begin{tikzpicture}[node distance = 2cm]
  \tikzstyle{process} = [rectangle, minimum width=3cm, minimum height=1cm, text centered, draw=black, fill=orange!30]
  \tikzstyle{arrow} = [thick,->,>=stealth]
  \node (r) [process, xshift=2cm] {Rumen};
  \node (a) [process, right of=r, xshift=2cm] {Abomasum};
  \node (d) [process, right of=a, xshift=2cm] {Duodemum};
  \node (f) [process, right of=d, xshift=2cm]  {Feces};
  \draw [arrow] (r) -- (a);
  \draw [arrow] (a) -- (d);
  \draw [arrow] (d) -- (f);
\end{tikzpicture}

\subsubsection{The experiment}\label{the-experiment}

The digestive process is most observable at the beginning and at the
end, we can observer and control what goes in and what comes out.

\subsubsection{The model}\label{the-model}

Suppose that at \(t = 0\) a sheep is fed an amount \(R\) of food which
goes immediately into its rumen. This food will pass gradually from the
rumen through the abomasum into the duodenum. At any later time \(t\) we
shall define:

\(r(t)\) = the amount of food still in the rumen; \(a(t)\) = the amount
in the abomasum; \(d(t)\) = the amount which by then has arrived in the
duodenum.

So \(r(0) = R\), \(a(0)\) = \(d(0)\) = 0, and, for all t\textgreater{}0,
r(t) + a(t) + d(t) = R.

\subsubsection{The assumptions}\label{the-assumptions}

Two assumptions are made: (A) Food moves out of the rumen at a rate
proportional to the amount of food in the rumen. Mathematically this
says:

\(r'(t) = -k_1r(t)\)

where k\_1 is a positive proportionality constant.

\begin{enumerate}
\def\labelenumi{(\Alph{enumi})}
\setcounter{enumi}{1}
\tightlist
\item
  Food moves out of the abomasum at a rate proportional to the amout of
  food in the abomasum. Since at the same time food is moving into the
  abomasum at the rate given by Equation (2), the assumption says
\end{enumerate}

\begin{enumerate}
\def\labelenumi{(\arabic{enumi})}
\setcounter{enumi}{2}
\tightlist
\item
  \(a'(t) = k_1r(t)-k_2a(t)\)
\end{enumerate}


\end{document}
