\documentclass[]{article}
\usepackage{lmodern}
\usepackage{amssymb,amsmath}
\usepackage{ifxetex,ifluatex}
\usepackage{fixltx2e} % provides \textsubscript
\ifnum 0\ifxetex 1\fi\ifluatex 1\fi=0 % if pdftex
  \usepackage[T1]{fontenc}
  \usepackage[utf8]{inputenc}
\else % if luatex or xelatex
  \ifxetex
    \usepackage{mathspec}
  \else
    \usepackage{fontspec}
  \fi
  \defaultfontfeatures{Ligatures=TeX,Scale=MatchLowercase}
\fi
% use upquote if available, for straight quotes in verbatim environments
\IfFileExists{upquote.sty}{\usepackage{upquote}}{}
% use microtype if available
\IfFileExists{microtype.sty}{%
\usepackage{microtype}
\UseMicrotypeSet[protrusion]{basicmath} % disable protrusion for tt fonts
}{}
\usepackage[margin=1in]{geometry}
\usepackage{hyperref}
\hypersetup{unicode=true,
            pdftitle={Finall\_all\_609},
            pdfauthor={Daina Bouquin, Christophe Hunt, Christina Taylor},
            pdfborder={0 0 0},
            breaklinks=true}
\urlstyle{same}  % don't use monospace font for urls
\usepackage{color}
\usepackage{fancyvrb}
\newcommand{\VerbBar}{|}
\newcommand{\VERB}{\Verb[commandchars=\\\{\}]}
\DefineVerbatimEnvironment{Highlighting}{Verbatim}{commandchars=\\\{\}}
% Add ',fontsize=\small' for more characters per line
\usepackage{framed}
\definecolor{shadecolor}{RGB}{248,248,248}
\newenvironment{Shaded}{\begin{snugshade}}{\end{snugshade}}
\newcommand{\KeywordTok}[1]{\textcolor[rgb]{0.13,0.29,0.53}{\textbf{#1}}}
\newcommand{\DataTypeTok}[1]{\textcolor[rgb]{0.13,0.29,0.53}{#1}}
\newcommand{\DecValTok}[1]{\textcolor[rgb]{0.00,0.00,0.81}{#1}}
\newcommand{\BaseNTok}[1]{\textcolor[rgb]{0.00,0.00,0.81}{#1}}
\newcommand{\FloatTok}[1]{\textcolor[rgb]{0.00,0.00,0.81}{#1}}
\newcommand{\ConstantTok}[1]{\textcolor[rgb]{0.00,0.00,0.00}{#1}}
\newcommand{\CharTok}[1]{\textcolor[rgb]{0.31,0.60,0.02}{#1}}
\newcommand{\SpecialCharTok}[1]{\textcolor[rgb]{0.00,0.00,0.00}{#1}}
\newcommand{\StringTok}[1]{\textcolor[rgb]{0.31,0.60,0.02}{#1}}
\newcommand{\VerbatimStringTok}[1]{\textcolor[rgb]{0.31,0.60,0.02}{#1}}
\newcommand{\SpecialStringTok}[1]{\textcolor[rgb]{0.31,0.60,0.02}{#1}}
\newcommand{\ImportTok}[1]{#1}
\newcommand{\CommentTok}[1]{\textcolor[rgb]{0.56,0.35,0.01}{\textit{#1}}}
\newcommand{\DocumentationTok}[1]{\textcolor[rgb]{0.56,0.35,0.01}{\textbf{\textit{#1}}}}
\newcommand{\AnnotationTok}[1]{\textcolor[rgb]{0.56,0.35,0.01}{\textbf{\textit{#1}}}}
\newcommand{\CommentVarTok}[1]{\textcolor[rgb]{0.56,0.35,0.01}{\textbf{\textit{#1}}}}
\newcommand{\OtherTok}[1]{\textcolor[rgb]{0.56,0.35,0.01}{#1}}
\newcommand{\FunctionTok}[1]{\textcolor[rgb]{0.00,0.00,0.00}{#1}}
\newcommand{\VariableTok}[1]{\textcolor[rgb]{0.00,0.00,0.00}{#1}}
\newcommand{\ControlFlowTok}[1]{\textcolor[rgb]{0.13,0.29,0.53}{\textbf{#1}}}
\newcommand{\OperatorTok}[1]{\textcolor[rgb]{0.81,0.36,0.00}{\textbf{#1}}}
\newcommand{\BuiltInTok}[1]{#1}
\newcommand{\ExtensionTok}[1]{#1}
\newcommand{\PreprocessorTok}[1]{\textcolor[rgb]{0.56,0.35,0.01}{\textit{#1}}}
\newcommand{\AttributeTok}[1]{\textcolor[rgb]{0.77,0.63,0.00}{#1}}
\newcommand{\RegionMarkerTok}[1]{#1}
\newcommand{\InformationTok}[1]{\textcolor[rgb]{0.56,0.35,0.01}{\textbf{\textit{#1}}}}
\newcommand{\WarningTok}[1]{\textcolor[rgb]{0.56,0.35,0.01}{\textbf{\textit{#1}}}}
\newcommand{\AlertTok}[1]{\textcolor[rgb]{0.94,0.16,0.16}{#1}}
\newcommand{\ErrorTok}[1]{\textcolor[rgb]{0.64,0.00,0.00}{\textbf{#1}}}
\newcommand{\NormalTok}[1]{#1}
\usepackage{longtable,booktabs}
\usepackage{graphicx,grffile}
\makeatletter
\def\maxwidth{\ifdim\Gin@nat@width>\linewidth\linewidth\else\Gin@nat@width\fi}
\def\maxheight{\ifdim\Gin@nat@height>\textheight\textheight\else\Gin@nat@height\fi}
\makeatother
% Scale images if necessary, so that they will not overflow the page
% margins by default, and it is still possible to overwrite the defaults
% using explicit options in \includegraphics[width, height, ...]{}
\setkeys{Gin}{width=\maxwidth,height=\maxheight,keepaspectratio}
\IfFileExists{parskip.sty}{%
\usepackage{parskip}
}{% else
\setlength{\parindent}{0pt}
\setlength{\parskip}{6pt plus 2pt minus 1pt}
}
\setlength{\emergencystretch}{3em}  % prevent overfull lines
\providecommand{\tightlist}{%
  \setlength{\itemsep}{0pt}\setlength{\parskip}{0pt}}
\setcounter{secnumdepth}{5}
% Redefines (sub)paragraphs to behave more like sections
\ifx\paragraph\undefined\else
\let\oldparagraph\paragraph
\renewcommand{\paragraph}[1]{\oldparagraph{#1}\mbox{}}
\fi
\ifx\subparagraph\undefined\else
\let\oldsubparagraph\subparagraph
\renewcommand{\subparagraph}[1]{\oldsubparagraph{#1}\mbox{}}
\fi
\usepackage{relsize}
\usepackage{setspace}
\usepackage{amsmath,amsfonts,amsthm}
\usepackage[sfdefault]{roboto}
\usepackage[T1]{fontenc}
\usepackage{float}
\usepackage{multirow}
\usepackage{mathtools}
\usepackage{tikz}
\usetikzlibrary{shapes,arrows}

%%% Use protect on footnotes to avoid problems with footnotes in titles
\let\rmarkdownfootnote\footnote%
\def\footnote{\protect\rmarkdownfootnote}

%%% Change title format to be more compact
\usepackage{titling}

% Create subtitle command for use in maketitle
\newcommand{\subtitle}[1]{
  \posttitle{
    \begin{center}\large#1\end{center}
    }
}

\setlength{\droptitle}{-2em}
  \title{Finall\_all\_609}
  \pretitle{\vspace{\droptitle}\centering\huge}
  \posttitle{\par}
  \author{Daina Bouquin, Christophe Hunt, Christina Taylor}
  \preauthor{\centering\large\emph}
  \postauthor{\par}
  \predate{\centering\large\emph}
  \postdate{\par}
  \date{May 15, 2017}

\begin{document}
\maketitle

{
\setcounter{tocdepth}{2}
\tableofcontents
}
\subsubsection{Textbook Projects from:}\label{textbook-projects-from}

Giordano, F. R., Fox, W. P., \& Horton, S. B. (2014). \emph{A first
course in mathematical modeling}. Australia: Brooks/Cole.

\subsection{Textbook Part I:}\label{textbook-part-i}

A farmer has 30 acres on which to grow tomatoes and corn. Each 100
bushels of tomatoes require 1000 gallons of water and 5 acres of land.
Each 100 bushels of corn require 6000 gallons of water and 2.5 acres of
land. Labor costs are \$1 per bushel for both corn and tomatoes. The
farm has available 30,000 gallons of water and \$750 in capital. He
knows he cannot sell more than 500 bushels of tomatoes and 475 bushels
of corn. He estimates a profit of \$2 on each bushel of tomatoes and \$3
on each bushel of corn.

\begin{enumerate}
\def\labelenumi{(\alph{enumi})}
\tightlist
\item
  How many bushels should he raise to maximize profits?
\end{enumerate}

Let \(t\) = bushels of tomatoes (in hundreds) \(c\) = bushels of corn
(in hundreds) \(w\) = gallons of water (in thousands) \(p\) = profit (in
dollars) \(a\) = land (in acres)

The farmer's problem can be modeled as Maximizing Profit:
\(200t + 300c\)

Subject to the following constraints: Non-negativity:
\(t \geq 0, c \geq 0\) Water: \(t + 6c \leq 30\) Land:
\(5t + 2.5c \leq 30\) Labor cost: \(t + c \leq 7.5\) Demand:
\(t \leq 5, c \leq 4.75\)

Apply linear programming solution

\begin{Shaded}
\begin{Highlighting}[]
\CommentTok{#create LP object with 2 decision variables}
\NormalTok{lprec =}\StringTok{ }\KeywordTok{make.lp}\NormalTok{(}\DecValTok{0}\NormalTok{,}\DecValTok{2}\NormalTok{)}
\CommentTok{#set objective function and direction (maximize)}
\KeywordTok{set.objfn}\NormalTok{(lprec,}\KeywordTok{c}\NormalTok{(}\DecValTok{200}\NormalTok{,}\DecValTok{300}\NormalTok{))}
\KeywordTok{lp.control}\NormalTok{(lprec,}\DataTypeTok{sense=}\StringTok{'max'}\NormalTok{)}
\end{Highlighting}
\end{Shaded}

\begin{verbatim}
## $anti.degen
## [1] "fixedvars" "stalling" 
## 
## $basis.crash
## [1] "none"
## 
## $bb.depthlimit
## [1] -50
## 
## $bb.floorfirst
## [1] "automatic"
## 
## $bb.rule
## [1] "pseudononint" "greedy"       "dynamic"      "rcostfixing" 
## 
## $break.at.first
## [1] FALSE
## 
## $break.at.value
## [1] 1e+30
## 
## $epsilon
##       epsb       epsd      epsel     epsint epsperturb   epspivot 
##      1e-10      1e-09      1e-12      1e-07      1e-05      2e-07 
## 
## $improve
## [1] "dualfeas" "thetagap"
## 
## $infinite
## [1] 1e+30
## 
## $maxpivot
## [1] 250
## 
## $mip.gap
## absolute relative 
##    1e-11    1e-11 
## 
## $negrange
## [1] -1e+06
## 
## $obj.in.basis
## [1] TRUE
## 
## $pivoting
## [1] "devex"    "adaptive"
## 
## $presolve
## [1] "none"
## 
## $scalelimit
## [1] 5
## 
## $scaling
## [1] "geometric"   "equilibrate" "integers"   
## 
## $sense
## [1] "maximize"
## 
## $simplextype
## [1] "dual"   "primal"
## 
## $timeout
## [1] 0
## 
## $verbose
## [1] "neutral"
\end{verbatim}

\begin{Shaded}
\begin{Highlighting}[]
\CommentTok{#add constraints}
\KeywordTok{add.constraint}\NormalTok{(lprec, }\KeywordTok{c}\NormalTok{(}\DecValTok{1}\NormalTok{,}\DecValTok{6}\NormalTok{), }\StringTok{"<="}\NormalTok{, }\DecValTok{30}\NormalTok{)}
\KeywordTok{add.constraint}\NormalTok{(lprec, }\KeywordTok{c}\NormalTok{(}\DecValTok{5}\NormalTok{,}\FloatTok{2.5}\NormalTok{), }\StringTok{"<="}\NormalTok{, }\DecValTok{30}\NormalTok{)}
\KeywordTok{add.constraint}\NormalTok{(lprec, }\KeywordTok{c}\NormalTok{(}\DecValTok{1}\NormalTok{,}\DecValTok{1}\NormalTok{), }\StringTok{"<="}\NormalTok{, }\FloatTok{7.5}\NormalTok{)}
\CommentTok{#set boundaries}
\KeywordTok{set.bounds}\NormalTok{(lprec, }\DataTypeTok{lower =} \KeywordTok{c}\NormalTok{(}\DecValTok{0}\NormalTok{,}\DecValTok{0}\NormalTok{), }\DataTypeTok{upper =} \KeywordTok{c}\NormalTok{(}\DecValTok{5}\NormalTok{,}\FloatTok{4.75}\NormalTok{))}

\CommentTok{#rename for printing}
\NormalTok{RowNames <-}\StringTok{ }\KeywordTok{c}\NormalTok{(}\StringTok{"water"}\NormalTok{, }\StringTok{"land"}\NormalTok{, }\StringTok{"labor"}\NormalTok{)}
\NormalTok{ColNames <-}\StringTok{ }\KeywordTok{c}\NormalTok{(}\StringTok{"tomatoes"}\NormalTok{,}\StringTok{"corn"}\NormalTok{)}
\KeywordTok{dimnames}\NormalTok{(lprec) <-}\StringTok{ }\KeywordTok{list}\NormalTok{(RowNames, ColNames)}
\CommentTok{#verify the model}
\NormalTok{lprec}
\end{Highlighting}
\end{Shaded}

\begin{verbatim}
## Model name: 
##           tomatoes      corn         
## Maximize       200       300         
## water            1         6  <=   30
## land             5       2.5  <=   30
## labor            1         1  <=  7.5
## Kind           Std       Std         
## Type          Real      Real         
## Upper            5      4.75         
## Lower            0         0
\end{verbatim}

\begin{Shaded}
\begin{Highlighting}[]
\CommentTok{#solve the model (0 means there is a feasible solution)}
\KeywordTok{solve}\NormalTok{(lprec)}
\end{Highlighting}
\end{Shaded}

\begin{verbatim}
## [1] 0
\end{verbatim}

\begin{Shaded}
\begin{Highlighting}[]
\CommentTok{#find extreme point}
\KeywordTok{get.variables}\NormalTok{(lprec)}
\end{Highlighting}
\end{Shaded}

\begin{verbatim}
## [1] 3.0 4.5
\end{verbatim}

\begin{Shaded}
\begin{Highlighting}[]
\CommentTok{#find maximize objective function value}
\KeywordTok{get.objective}\NormalTok{(lprec)}
\end{Highlighting}
\end{Shaded}

\begin{verbatim}
## [1] 1950
\end{verbatim}

We can see that if the farmer grows 300 bushels of tomatoes and 450
bushels of corn, he can maximize his profit at \$1950.

\begin{enumerate}
\def\labelenumi{(\alph{enumi})}
\setcounter{enumi}{1}
\tightlist
\item
  Assume the farmer has the oppertunity to sign a contract with grocery
  store to grow and deliver at least 300 bushels of tomatoes and 500
  bushels of corn. Should he sign the contract?
\end{enumerate}

The new demand constraint is now \(t \geq 3, c \geq 5\) Intuitively, we
can easily see that this condition is incompatible with the constraint
of labor costs. We can verify this using linear programming:

\begin{Shaded}
\begin{Highlighting}[]
\NormalTok{lprec =}\StringTok{ }\KeywordTok{make.lp}\NormalTok{(}\DecValTok{0}\NormalTok{,}\DecValTok{2}\NormalTok{)}
\CommentTok{#set objective function and direction (maximize)}
\KeywordTok{set.objfn}\NormalTok{(lprec,}\KeywordTok{c}\NormalTok{(}\DecValTok{200}\NormalTok{,}\DecValTok{300}\NormalTok{))}
\KeywordTok{lp.control}\NormalTok{(lprec,}\DataTypeTok{sense=}\StringTok{'max'}\NormalTok{)}
\end{Highlighting}
\end{Shaded}

\begin{verbatim}
## $anti.degen
## [1] "fixedvars" "stalling" 
## 
## $basis.crash
## [1] "none"
## 
## $bb.depthlimit
## [1] -50
## 
## $bb.floorfirst
## [1] "automatic"
## 
## $bb.rule
## [1] "pseudononint" "greedy"       "dynamic"      "rcostfixing" 
## 
## $break.at.first
## [1] FALSE
## 
## $break.at.value
## [1] 1e+30
## 
## $epsilon
##       epsb       epsd      epsel     epsint epsperturb   epspivot 
##      1e-10      1e-09      1e-12      1e-07      1e-05      2e-07 
## 
## $improve
## [1] "dualfeas" "thetagap"
## 
## $infinite
## [1] 1e+30
## 
## $maxpivot
## [1] 250
## 
## $mip.gap
## absolute relative 
##    1e-11    1e-11 
## 
## $negrange
## [1] -1e+06
## 
## $obj.in.basis
## [1] TRUE
## 
## $pivoting
## [1] "devex"    "adaptive"
## 
## $presolve
## [1] "none"
## 
## $scalelimit
## [1] 5
## 
## $scaling
## [1] "geometric"   "equilibrate" "integers"   
## 
## $sense
## [1] "maximize"
## 
## $simplextype
## [1] "dual"   "primal"
## 
## $timeout
## [1] 0
## 
## $verbose
## [1] "neutral"
\end{verbatim}

\begin{Shaded}
\begin{Highlighting}[]
\CommentTok{#add constraints}
\KeywordTok{add.constraint}\NormalTok{(lprec, }\KeywordTok{c}\NormalTok{(}\DecValTok{1}\NormalTok{,}\DecValTok{6}\NormalTok{), }\StringTok{"<="}\NormalTok{, }\DecValTok{30}\NormalTok{)}
\KeywordTok{add.constraint}\NormalTok{(lprec, }\KeywordTok{c}\NormalTok{(}\DecValTok{5}\NormalTok{,}\FloatTok{2.5}\NormalTok{), }\StringTok{"<="}\NormalTok{, }\DecValTok{30}\NormalTok{)}
\KeywordTok{add.constraint}\NormalTok{(lprec, }\KeywordTok{c}\NormalTok{(}\DecValTok{1}\NormalTok{,}\DecValTok{1}\NormalTok{), }\StringTok{"<="}\NormalTok{, }\FloatTok{7.5}\NormalTok{)}
\CommentTok{#set boundaries}
\KeywordTok{set.bounds}\NormalTok{(lprec, }\DataTypeTok{lower =} \KeywordTok{c}\NormalTok{(}\DecValTok{3}\NormalTok{,}\DecValTok{5}\NormalTok{))}

\CommentTok{#solve the model}
\KeywordTok{solve}\NormalTok{(lprec)}
\end{Highlighting}
\end{Shaded}

\begin{verbatim}
## [1] 2
\end{verbatim}

As shown, there is no feasible solution. The farmer should not sign the
contract, because realistically, he cannot deliver.

\begin{enumerate}
\def\labelenumi{(\alph{enumi})}
\setcounter{enumi}{2}
\tightlist
\item
  Assume the farmer can obtain an additional 10,000 gallons of water for
  \$50. Should he obtain the additional water?
\end{enumerate}

The new water constraint is now \(t + 6c \leq 40\) The farmer's profit
is \(200t + 300c - 50\)

\begin{Shaded}
\begin{Highlighting}[]
\NormalTok{lprec =}\StringTok{ }\KeywordTok{make.lp}\NormalTok{(}\DecValTok{0}\NormalTok{,}\DecValTok{2}\NormalTok{)}
\CommentTok{#set objective function and direction (maximize)}
\KeywordTok{set.objfn}\NormalTok{(lprec,}\KeywordTok{c}\NormalTok{(}\DecValTok{200}\NormalTok{,}\DecValTok{300}\NormalTok{))}
\KeywordTok{lp.control}\NormalTok{(lprec,}\DataTypeTok{sense=}\StringTok{'max'}\NormalTok{)}
\end{Highlighting}
\end{Shaded}

\begin{verbatim}
## $anti.degen
## [1] "fixedvars" "stalling" 
## 
## $basis.crash
## [1] "none"
## 
## $bb.depthlimit
## [1] -50
## 
## $bb.floorfirst
## [1] "automatic"
## 
## $bb.rule
## [1] "pseudononint" "greedy"       "dynamic"      "rcostfixing" 
## 
## $break.at.first
## [1] FALSE
## 
## $break.at.value
## [1] 1e+30
## 
## $epsilon
##       epsb       epsd      epsel     epsint epsperturb   epspivot 
##      1e-10      1e-09      1e-12      1e-07      1e-05      2e-07 
## 
## $improve
## [1] "dualfeas" "thetagap"
## 
## $infinite
## [1] 1e+30
## 
## $maxpivot
## [1] 250
## 
## $mip.gap
## absolute relative 
##    1e-11    1e-11 
## 
## $negrange
## [1] -1e+06
## 
## $obj.in.basis
## [1] TRUE
## 
## $pivoting
## [1] "devex"    "adaptive"
## 
## $presolve
## [1] "none"
## 
## $scalelimit
## [1] 5
## 
## $scaling
## [1] "geometric"   "equilibrate" "integers"   
## 
## $sense
## [1] "maximize"
## 
## $simplextype
## [1] "dual"   "primal"
## 
## $timeout
## [1] 0
## 
## $verbose
## [1] "neutral"
\end{verbatim}

\begin{Shaded}
\begin{Highlighting}[]
\CommentTok{#add constraints}
\KeywordTok{add.constraint}\NormalTok{(lprec, }\KeywordTok{c}\NormalTok{(}\DecValTok{1}\NormalTok{,}\DecValTok{6}\NormalTok{), }\StringTok{"<="}\NormalTok{, }\DecValTok{40}\NormalTok{)}
\KeywordTok{add.constraint}\NormalTok{(lprec, }\KeywordTok{c}\NormalTok{(}\DecValTok{5}\NormalTok{,}\FloatTok{2.5}\NormalTok{), }\StringTok{"<="}\NormalTok{, }\DecValTok{30}\NormalTok{)}
\KeywordTok{add.constraint}\NormalTok{(lprec, }\KeywordTok{c}\NormalTok{(}\DecValTok{1}\NormalTok{,}\DecValTok{1}\NormalTok{), }\StringTok{"<="}\NormalTok{, }\FloatTok{7.5}\NormalTok{)}
\CommentTok{#set boundaries}
\KeywordTok{set.bounds}\NormalTok{(lprec, }\DataTypeTok{lower =} \KeywordTok{c}\NormalTok{(}\DecValTok{0}\NormalTok{,}\DecValTok{0}\NormalTok{), }\DataTypeTok{upper =} \KeywordTok{c}\NormalTok{(}\DecValTok{5}\NormalTok{,}\FloatTok{4.75}\NormalTok{))}

\NormalTok{RowNames <-}\StringTok{ }\KeywordTok{c}\NormalTok{(}\StringTok{"water"}\NormalTok{, }\StringTok{"land"}\NormalTok{, }\StringTok{"labor"}\NormalTok{)}
\NormalTok{ColNames <-}\StringTok{ }\KeywordTok{c}\NormalTok{(}\StringTok{"tomatoes"}\NormalTok{,}\StringTok{"corn"}\NormalTok{)}
\KeywordTok{dimnames}\NormalTok{(lprec) <-}\StringTok{ }\KeywordTok{list}\NormalTok{(RowNames, ColNames)}

\CommentTok{#solve the model}
\KeywordTok{solve}\NormalTok{(lprec)}
\end{Highlighting}
\end{Shaded}

\begin{verbatim}
## [1] 0
\end{verbatim}

\begin{Shaded}
\begin{Highlighting}[]
\CommentTok{#find extreme point}
\KeywordTok{get.variables}\NormalTok{(lprec)}
\end{Highlighting}
\end{Shaded}

\begin{verbatim}
## [1] 2.75 4.75
\end{verbatim}

\begin{Shaded}
\begin{Highlighting}[]
\CommentTok{#find maximize profit}
\KeywordTok{get.objective}\NormalTok{(lprec) }\OperatorTok{-}\StringTok{ }\DecValTok{50}
\end{Highlighting}
\end{Shaded}

\begin{verbatim}
## [1] 1925
\end{verbatim}

If the farmer grows 275 bushels of tomatoes and 475 bushels of corn, he
can maximize his profit at \$1975. However, given the \$50 cost in
acquiring extra water, his net profit actually went down to \$1925.
Therefore, he should not obtain additional water.

\newpage

\subsection{Textbook part II: Chapter 5.3, Project
1}\label{textbook-part-ii-chapter-5.3-project-1}

Construct and perform a Monte Carlo simulation of blackjack as per the
following:

\begin{itemize}
\tightlist
\item
  Play 12 games (simulations) where each game lasts two decks. When the
  two decks are out, the round is completed using two fresh decks (that
  is the last round of the game). Everything is then reset for the start
  of the next game.\\
\item
  The dealer cannot see the players cards and vice versa.\\
\item
  The player wins 3 dollars with a winning hand.\\
\item
  The player loses 2 dollars with a losing hand.\\
\item
  No money is exchanged if there is no winner.

  \begin{itemize}
  \tightlist
  \item
    There is no winner when neither has gone bust and they stand at the
    same amount.\\
  \item
    If the dealer goes bust, the player automatically wins.\\
  \end{itemize}
\item
  The dealer strategy is to stand at 17 or above.\\
\item
  The player strategy is open and can be set as desired.
\end{itemize}

\paragraph{Create a two deck game}\label{create-a-two-deck-game}

The below functions create the game of blackjack from a dealt card, to a
hand, to a round, and to a full two-deck game. Game results are stored
as a vector: 1 = dealer win; 2 = player win; 0 = no winner. The
resulting vector is used to calculate player winnings/losses.

\begin{Shaded}
\begin{Highlighting}[]
\CommentTok{# Create a hand for the dealer. }
\NormalTok{deal_card <-}\StringTok{ }\ControlFlowTok{function}\NormalTok{(d_deck,decks_}\DecValTok{2}\NormalTok{)\{}
\NormalTok{  flag_last_round <-}\StringTok{ }\DecValTok{0}
  \ControlFlowTok{if}\NormalTok{ (}\KeywordTok{length}\NormalTok{(d_deck) }\OperatorTok{==}\StringTok{ }\DecValTok{0}\NormalTok{) \{}
\NormalTok{    flag_last_round <-}\StringTok{ }\DecValTok{1}
\NormalTok{    d_deck <-}\StringTok{ }\NormalTok{decks_}\DecValTok{2}
\NormalTok{  \}}
\NormalTok{  d_card <-}\StringTok{ }\KeywordTok{sample}\NormalTok{(d_deck,}\DecValTok{1}\NormalTok{,}\OtherTok{FALSE}\NormalTok{)}
\NormalTok{  d_deck <-}\StringTok{ }\NormalTok{d_deck [}\OperatorTok{!}\NormalTok{d_deck }\OperatorTok\StringTok{ }\NormalTok{d_card }\OperatorTok{|}\StringTok{ }\KeywordTok{duplicated}\NormalTok{(d_deck)]}
  \KeywordTok{return}\NormalTok{(}\KeywordTok{list}\NormalTok{(d_card,d_deck,flag_last_round))}
\NormalTok{\}}
\NormalTok{dealer_hand <-}\StringTok{ }\ControlFlowTok{function}\NormalTok{(dh_deck,decks_}\DecValTok{2}\NormalTok{) \{}
\NormalTok{  dh_hand <-}\StringTok{ }\DecValTok{0}
\NormalTok{  dh_last_round_flag <-}\StringTok{ }\DecValTok{0}
\NormalTok{  dh_count <-}\StringTok{ }\DecValTok{1}
  \ControlFlowTok{while}\NormalTok{ (}\KeywordTok{sum}\NormalTok{(dh_hand) }\OperatorTok{<}\StringTok{ }\DecValTok{17}\NormalTok{) \{}
\NormalTok{    dh_deal <-}\StringTok{ }\KeywordTok{deal_card}\NormalTok{(dh_deck,decks_}\DecValTok{2}\NormalTok{)}
\NormalTok{    dh_hand[dh_count] <-}\StringTok{ }\NormalTok{dh_deal[[}\DecValTok{1}\NormalTok{]]}
\NormalTok{    dh_deck <-}\StringTok{ }\NormalTok{dh_deal[[}\DecValTok{2}\NormalTok{]]}
    \ControlFlowTok{if}\NormalTok{ (dh_deal[[}\DecValTok{3}\NormalTok{]] }\OperatorTok{==}\StringTok{ }\DecValTok{1}\NormalTok{) \{dh_last_round_flag <-}\StringTok{ }\NormalTok{dh_deal[[}\DecValTok{3}\NormalTok{]]\}}
\NormalTok{    dh_count <-}\StringTok{ }\NormalTok{dh_count}\OperatorTok{+}\DecValTok{1} \CommentTok{# ace = 1 if hand > 21}
    \ControlFlowTok{if}\NormalTok{(}\KeywordTok{sum}\NormalTok{(dh_hand) }\OperatorTok{>}\StringTok{ }\DecValTok{21}\NormalTok{) \{}
      \ControlFlowTok{if}\NormalTok{(}\DecValTok{11} \OperatorTok\StringTok{ }\NormalTok{dh_hand) \{}
\NormalTok{        dh_hand[}\KeywordTok{match}\NormalTok{(}\DecValTok{11}\NormalTok{,dh_hand)] <-}\StringTok{ }\DecValTok{1}
\NormalTok{      \}}
\NormalTok{    \}}
\NormalTok{  \}    }
  \KeywordTok{return}\NormalTok{(}\KeywordTok{list}\NormalTok{(dh_hand,dh_deck,dh_last_round_flag))}
\NormalTok{\}}

\CommentTok{# Create player hand }
\NormalTok{player_hand <-}\StringTok{ }\ControlFlowTok{function}\NormalTok{(ph_deck,decks_}\DecValTok{2}\NormalTok{,threshold_p) \{}
\NormalTok{  ph_hand <-}\StringTok{ }\DecValTok{0}
\NormalTok{  ph_count <-}\StringTok{ }\DecValTok{1}
\NormalTok{  ph_last_round_flag <-}\StringTok{ }\DecValTok{0}
  \ControlFlowTok{while}\NormalTok{ (}\KeywordTok{sum}\NormalTok{(ph_hand) }\OperatorTok{<}\StringTok{ }\NormalTok{threshold_p) \{}
\NormalTok{    ph_deal <-}\StringTok{ }\KeywordTok{deal_card}\NormalTok{(ph_deck,decks_}\DecValTok{2}\NormalTok{)}
\NormalTok{    ph_hand[ph_count] <-}\StringTok{ }\NormalTok{ph_deal[[}\DecValTok{1}\NormalTok{]]}
\NormalTok{    ph_deck <-}\StringTok{ }\NormalTok{ph_deal[[}\DecValTok{2}\NormalTok{]]}
    \ControlFlowTok{if}\NormalTok{ (ph_deal[[}\DecValTok{3}\NormalTok{]] }\OperatorTok{==}\StringTok{ }\DecValTok{1}\NormalTok{) \{ph_last_round_flag <-}\StringTok{ }\NormalTok{ph_deal[[}\DecValTok{3}\NormalTok{]]\}}
\NormalTok{    ph_count <-}\StringTok{ }\NormalTok{ph_count}\OperatorTok{+}\DecValTok{1} \CommentTok{# ace = 1 if hand > 21}
    \ControlFlowTok{if}\NormalTok{(}\KeywordTok{sum}\NormalTok{(ph_hand) }\OperatorTok{>}\StringTok{ }\DecValTok{21}\NormalTok{) \{}
      \ControlFlowTok{if}\NormalTok{(}\DecValTok{11} \OperatorTok\StringTok{ }\NormalTok{ph_hand) \{}
\NormalTok{        ph_hand[}\KeywordTok{match}\NormalTok{(}\DecValTok{11}\NormalTok{,ph_hand)] <-}\StringTok{ }\DecValTok{1}
\NormalTok{      \}}
\NormalTok{    \}}
\NormalTok{  \}  }
  \KeywordTok{return}\NormalTok{(}\KeywordTok{list}\NormalTok{(ph_hand,ph_deck,ph_last_round_flag))}
\NormalTok{\}}

\CommentTok{# Create a game }
\NormalTok{game <-}\StringTok{ }\ControlFlowTok{function}\NormalTok{(game_deck,decks_}\DecValTok{2}\NormalTok{,threshold_p) \{ }
  \CommentTok{#dealer hand}
\NormalTok{  d_result <-}\StringTok{ }\KeywordTok{dealer_hand}\NormalTok{(game_deck,decks_}\DecValTok{2}\NormalTok{)}
\NormalTok{  d_hand <-}\StringTok{ }\NormalTok{d_result[[}\DecValTok{1}\NormalTok{]]}
\NormalTok{  game_deck <-}\StringTok{ }\NormalTok{d_result[[}\DecValTok{2}\NormalTok{]]}
\NormalTok{  last_round_flag_d <-}\StringTok{ }\NormalTok{d_result[[}\DecValTok{3}\NormalTok{]]}
  \CommentTok{#player hand}
\NormalTok{  p_result <-}\StringTok{ }\KeywordTok{player_hand}\NormalTok{(game_deck,decks_}\DecValTok{2}\NormalTok{,threshold_p)}
\NormalTok{  p_hand <-}\StringTok{ }\NormalTok{p_result[[}\DecValTok{1}\NormalTok{]]}
\NormalTok{  game_deck <-}\StringTok{ }\NormalTok{p_result[[}\DecValTok{2}\NormalTok{]]}
\NormalTok{  last_round_flag_p <-}\StringTok{ }\NormalTok{p_result[[}\DecValTok{3}\NormalTok{]]}
  \ControlFlowTok{if}\NormalTok{ (last_round_flag_d }\OperatorTok{==}\StringTok{ }\DecValTok{1} \OperatorTok{||}\StringTok{ }\NormalTok{last_round_flag_p }\OperatorTok{==}\StringTok{ }\DecValTok{1}\NormalTok{) \{}
\NormalTok{    game_last_round_flag <-}\StringTok{ }\DecValTok{1}
\NormalTok{  \} }\ControlFlowTok{else}\NormalTok{ \{}
\NormalTok{    game_last_round_flag <-}\StringTok{ }\DecValTok{0}
\NormalTok{  \}}
  \KeywordTok{return}\NormalTok{(}\KeywordTok{list}\NormalTok{(d_hand,p_hand,game_deck,game_last_round_flag))}
\NormalTok{\}}

\CommentTok{# Go through two decks- create dealer and player hand each round }
\NormalTok{play_decks_}\DecValTok{2}\NormalTok{ <-}\StringTok{ }\ControlFlowTok{function}\NormalTok{(p2d_deck,decks_}\DecValTok{2}\NormalTok{,threshold_p) \{}
\NormalTok{  round_counter <-}\StringTok{ }\DecValTok{1}
\NormalTok{  p2d_last_round_flag <-}\StringTok{ }\DecValTok{0} 
\NormalTok{  result_list.names <-}\StringTok{ }\KeywordTok{c}\NormalTok{(}\StringTok{"Dealer_Hand"}\NormalTok{, }\StringTok{"Player_Hand"}\NormalTok{)}
\NormalTok{  result_list <-}\StringTok{ }\KeywordTok{vector}\NormalTok{(}\StringTok{"list"}\NormalTok{, }\KeywordTok{length}\NormalTok{(result_list.names))}
  \KeywordTok{names}\NormalTok{(result_list) <-}\StringTok{ }\NormalTok{result_list.names}
  \ControlFlowTok{while}\NormalTok{ (p2d_last_round_flag }\OperatorTok{!=}\StringTok{ }\DecValTok{1}\NormalTok{) \{}
\NormalTok{    round_result <-}\StringTok{ }\KeywordTok{game}\NormalTok{(p2d_deck,decks_}\DecValTok{2}\NormalTok{,threshold_p)}
\NormalTok{    result_list}\OperatorTok{$}\NormalTok{Dealer_Hand[round_counter] <-}\StringTok{ }\KeywordTok{list}\NormalTok{(round_result[[}\DecValTok{1}\NormalTok{]])}
\NormalTok{    result_list}\OperatorTok{$}\NormalTok{Player_Hand[round_counter] <-}\StringTok{ }\KeywordTok{list}\NormalTok{(round_result[[}\DecValTok{2}\NormalTok{]])}
\NormalTok{    p2d_deck <-}\StringTok{ }\NormalTok{round_result[[}\DecValTok{3}\NormalTok{]]}
\NormalTok{    p2d_last_round_flag <-}\StringTok{ }\NormalTok{round_result[[}\DecValTok{4}\NormalTok{]]}
\NormalTok{    round_counter <-}\StringTok{ }\NormalTok{round_counter}\OperatorTok{+}\DecValTok{1}
\NormalTok{  \}}
  \KeywordTok{return}\NormalTok{(result_list)  }
\NormalTok{\}}

\CommentTok{# Takes in list containing all hands after above running through two decks. }
\NormalTok{results_vec <-}\StringTok{ }\ControlFlowTok{function}\NormalTok{(results_set) \{}
\NormalTok{  winner <-}\StringTok{ }\KeywordTok{numeric}\NormalTok{(}\KeywordTok{length}\NormalTok{(results_set}\OperatorTok{$}\NormalTok{Dealer_Hand))}
  \ControlFlowTok{for}\NormalTok{ (i }\ControlFlowTok{in} \DecValTok{1}\OperatorTok{:}\KeywordTok{length}\NormalTok{(results_set}\OperatorTok{$}\NormalTok{Dealer_Hand)) \{}
\NormalTok{    d_Hand <-}\StringTok{ }\KeywordTok{sum}\NormalTok{(results_set}\OperatorTok{$}\NormalTok{Dealer_Hand[[i]])}
\NormalTok{    p_Hand <-}\StringTok{ }\KeywordTok{sum}\NormalTok{(results_set}\OperatorTok{$}\NormalTok{Player_Hand[[i]])}
    \ControlFlowTok{if}\NormalTok{(d_Hand }\OperatorTok{<}\StringTok{ }\DecValTok{22}\NormalTok{) \{}
      \ControlFlowTok{if}\NormalTok{(p_Hand }\OperatorTok{<}\StringTok{ }\DecValTok{22}\NormalTok{) \{}
        \ControlFlowTok{if}\NormalTok{(d_Hand }\OperatorTok{>}\StringTok{ }\NormalTok{p_Hand) \{}
\NormalTok{          winner[i] <-}\StringTok{ }\DecValTok{1}  
\NormalTok{        \} }\ControlFlowTok{else} \ControlFlowTok{if}\NormalTok{ (d_Hand }\OperatorTok{<}\StringTok{ }\NormalTok{p_Hand) \{}
\NormalTok{            winner[i] <-}\StringTok{ }\DecValTok{2}
\NormalTok{        \} }\ControlFlowTok{else} \ControlFlowTok{if}\NormalTok{ (d_Hand }\OperatorTok{==}\StringTok{ }\NormalTok{p_Hand) \{}
\NormalTok{            winner[i] <-}\StringTok{ }\DecValTok{0}
\NormalTok{        \}}
\NormalTok{      \} }\ControlFlowTok{else} \ControlFlowTok{if}\NormalTok{(p_Hand }\OperatorTok{>=}\StringTok{ }\DecValTok{22}\NormalTok{) \{}
\NormalTok{          winner[i] <-}\StringTok{ }\DecValTok{1}     
\NormalTok{      \}}
\NormalTok{    \} }\ControlFlowTok{else} \ControlFlowTok{if}\NormalTok{(d_Hand }\OperatorTok{>=}\StringTok{ }\DecValTok{22}\NormalTok{)\{}
\NormalTok{            winner[i] <-}\StringTok{ }\DecValTok{2} \CommentTok{# if dealer goes bust player wins}
\NormalTok{      \} }
\NormalTok{  \}}
  \KeywordTok{return}\NormalTok{(winner)}
\NormalTok{\}}

\CommentTok{# Calculates the player winnings with assumptions: }
\CommentTok{# player bets $2/hand    }
\CommentTok{# player wins = player receives $3    }
\CommentTok{# player loss = player loses $2 bet    }
\CommentTok{# tie = $0 exchanged    }
\CommentTok{# dealer goes bust = win for player}
\NormalTok{calc_player_win <-}\StringTok{ }\ControlFlowTok{function}\NormalTok{(results) \{}
\NormalTok{  winnings <-}\StringTok{ }\KeywordTok{numeric}\NormalTok{(}\KeywordTok{length}\NormalTok{(results))}
  \ControlFlowTok{for}\NormalTok{(i }\ControlFlowTok{in} \DecValTok{1}\OperatorTok{:}\KeywordTok{length}\NormalTok{(results)) \{}
    \ControlFlowTok{if}\NormalTok{ (results[i] }\OperatorTok{==}\StringTok{ }\DecValTok{0}\NormalTok{) winnings[i] <-}\StringTok{ }\DecValTok{0}
    \ControlFlowTok{if}\NormalTok{ (results[i] }\OperatorTok{==}\StringTok{ }\DecValTok{1}\NormalTok{) winnings[i] <-}\StringTok{ }\OperatorTok{-}\DecValTok{2}
    \ControlFlowTok{if}\NormalTok{ (results[i] }\OperatorTok{==}\StringTok{ }\DecValTok{2}\NormalTok{) winnings[i] <-}\StringTok{ }\DecValTok{3}      
\NormalTok{  \}}
  \KeywordTok{return}\NormalTok{(winnings)}
\NormalTok{\}}

\CommentTok{# Play a full two deck game, calculate winnings/losses}
\NormalTok{play_deck2 <-}\StringTok{ }\ControlFlowTok{function}\NormalTok{(pd2_deck,threshold_p)\{}
  \CommentTok{#create list of results per round after running two decks game}
\NormalTok{  one_set <-}\StringTok{ }\KeywordTok{play_decks_2}\NormalTok{(pd2_deck,pd2_deck,threshold_p)}
  \CommentTok{#create vector for results of each hand}
\NormalTok{  outcome_vec <-}\StringTok{ }\KeywordTok{results_vec}\NormalTok{(one_set)}
\NormalTok{  player_winnings <-}\StringTok{ }\DecValTok{0}
\NormalTok{  player_winnings <-}\StringTok{ }\KeywordTok{calc_player_win}\NormalTok{(outcome_vec)}
  \KeywordTok{return}\NormalTok{(}\KeywordTok{sum}\NormalTok{(player_winnings))}
\NormalTok{\}}

\CommentTok{# Create function to run the simulation for two deck games }
\CommentTok{# Takes two decks and number of simulations as params }
\CommentTok{# Each run uses player strategies (stand at 16 to 20); dealer stands at 17}
\NormalTok{build_output <-}\StringTok{ }\ControlFlowTok{function}\NormalTok{(out_deck, runs) \{}
\NormalTok{  count_obs <-}\StringTok{ }\KeywordTok{numeric}\NormalTok{(runs)}
\NormalTok{  vec16 <-}\StringTok{ }\KeywordTok{numeric}\NormalTok{(runs)}
\NormalTok{  vec17 <-}\StringTok{ }\KeywordTok{numeric}\NormalTok{(runs)}
\NormalTok{  vec18 <-}\StringTok{ }\KeywordTok{numeric}\NormalTok{(runs)}
\NormalTok{  vec19 <-}\StringTok{ }\KeywordTok{numeric}\NormalTok{(runs)}
\NormalTok{  vec20 <-}\StringTok{ }\KeywordTok{numeric}\NormalTok{(runs)}
  \ControlFlowTok{for}\NormalTok{ (i }\ControlFlowTok{in} \DecValTok{1}\OperatorTok{:}\NormalTok{runs)\{}
\NormalTok{    count_obs[i] <-}\StringTok{ }\NormalTok{i}
\NormalTok{    vec16[i] <-}\StringTok{ }\KeywordTok{play_deck2}\NormalTok{(out_deck,}\DecValTok{16}\NormalTok{)}
\NormalTok{    vec17[i] <-}\StringTok{ }\KeywordTok{play_deck2}\NormalTok{(out_deck,}\DecValTok{17}\NormalTok{)}
\NormalTok{    vec18[i] <-}\StringTok{ }\KeywordTok{play_deck2}\NormalTok{(out_deck,}\DecValTok{18}\NormalTok{)}
\NormalTok{    vec19[i] <-}\StringTok{ }\KeywordTok{play_deck2}\NormalTok{(out_deck,}\DecValTok{19}\NormalTok{)}
\NormalTok{    vec20[i] <-}\StringTok{ }\KeywordTok{play_deck2}\NormalTok{(out_deck,}\DecValTok{20}\NormalTok{)}
\NormalTok{  \}}
\NormalTok{  output <-}\StringTok{ }\KeywordTok{data.frame}\NormalTok{(}\StringTok{"count_obs"}\NormalTok{=}\StringTok{ }\NormalTok{count_obs,}
                     \StringTok{"stand_16"}\NormalTok{=vec16,}\StringTok{"stand_17"}\NormalTok{=vec17,}
                     \StringTok{"stand_18"}\NormalTok{=vec18,}\StringTok{"stand_19"}\NormalTok{=vec19,}
                     \StringTok{"stand_20"}\NormalTok{=vec20)}
  \KeywordTok{return}\NormalTok{(output)  }
\NormalTok{\}}
\end{Highlighting}
\end{Shaded}

\paragraph{12 Simulations}\label{simulations}

\begin{Shaded}
\begin{Highlighting}[]
\KeywordTok{set.seed}\NormalTok{(}\DecValTok{2345}\NormalTok{)}
\CommentTok{# Create set of two decks, simulate 12 two-deck games; run sim with diff strategies}

\CommentTok{# vector for two decks}
\NormalTok{two_deck <-}\StringTok{ }\KeywordTok{c}\NormalTok{(}\DecValTok{2}\NormalTok{,}\DecValTok{2}\NormalTok{,}\DecValTok{2}\NormalTok{,}\DecValTok{2}\NormalTok{,}\DecValTok{3}\NormalTok{,}\DecValTok{3}\NormalTok{,}\DecValTok{3}\NormalTok{,}\DecValTok{3}\NormalTok{,}\DecValTok{4}\NormalTok{,}\DecValTok{4}\NormalTok{,}\DecValTok{4}\NormalTok{,}\DecValTok{4}\NormalTok{,}\DecValTok{5}\NormalTok{,}\DecValTok{5}\NormalTok{,}\DecValTok{5}\NormalTok{,}\DecValTok{5}\NormalTok{,}\DecValTok{6}\NormalTok{,}\DecValTok{6}\NormalTok{,}\DecValTok{6}\NormalTok{,}\DecValTok{6}\NormalTok{,}\DecValTok{7}\NormalTok{,}\DecValTok{7}\NormalTok{,}\DecValTok{7}\NormalTok{,}\DecValTok{7}\NormalTok{,}\DecValTok{8}\NormalTok{,}\DecValTok{8}\NormalTok{,}\DecValTok{8}\NormalTok{,}\DecValTok{8}\NormalTok{,}
              \DecValTok{9}\NormalTok{,}\DecValTok{9}\NormalTok{,}\DecValTok{9}\NormalTok{,}\DecValTok{9}\NormalTok{,}\DecValTok{10}\NormalTok{,}\DecValTok{10}\NormalTok{,}\DecValTok{10}\NormalTok{,}\DecValTok{10}\NormalTok{,}\DecValTok{10}\NormalTok{,}\DecValTok{10}\NormalTok{,}\DecValTok{10}\NormalTok{,}\DecValTok{10}\NormalTok{,}\DecValTok{10}\NormalTok{,}\DecValTok{10}\NormalTok{,}\DecValTok{10}\NormalTok{,}\DecValTok{10}\NormalTok{,}\DecValTok{10}\NormalTok{,}\DecValTok{10}\NormalTok{,}\DecValTok{10}\NormalTok{,}\DecValTok{10}\NormalTok{,}
              \DecValTok{11}\NormalTok{,}\DecValTok{11}\NormalTok{,}\DecValTok{11}\NormalTok{,}\DecValTok{11}\NormalTok{,}\DecValTok{2}\NormalTok{,}\DecValTok{2}\NormalTok{,}\DecValTok{2}\NormalTok{,}\DecValTok{2}\NormalTok{,}\DecValTok{3}\NormalTok{,}\DecValTok{3}\NormalTok{,}\DecValTok{3}\NormalTok{,}\DecValTok{3}\NormalTok{,}\DecValTok{4}\NormalTok{,}\DecValTok{4}\NormalTok{,}\DecValTok{4}\NormalTok{,}\DecValTok{4}\NormalTok{,}\DecValTok{5}\NormalTok{,}\DecValTok{5}\NormalTok{,}\DecValTok{5}\NormalTok{,}\DecValTok{5}\NormalTok{,}\DecValTok{6}\NormalTok{,}\DecValTok{6}\NormalTok{,}\DecValTok{6}\NormalTok{,}\DecValTok{6}\NormalTok{,}
              \DecValTok{7}\NormalTok{,}\DecValTok{7}\NormalTok{,}\DecValTok{7}\NormalTok{,}\DecValTok{7}\NormalTok{,}\DecValTok{8}\NormalTok{,}\DecValTok{8}\NormalTok{,}\DecValTok{8}\NormalTok{,}\DecValTok{8}\NormalTok{,}\DecValTok{9}\NormalTok{,}\DecValTok{9}\NormalTok{,}\DecValTok{9}\NormalTok{,}\DecValTok{9}\NormalTok{,}\DecValTok{10}\NormalTok{,}\DecValTok{10}\NormalTok{,}\DecValTok{10}\NormalTok{,}\DecValTok{10}\NormalTok{,}\DecValTok{10}\NormalTok{,}\DecValTok{10}\NormalTok{,}\DecValTok{10}\NormalTok{,}\DecValTok{10}\NormalTok{,}\DecValTok{10}\NormalTok{,}\DecValTok{10}\NormalTok{,}\DecValTok{10}\NormalTok{,}
              \DecValTok{10}\NormalTok{,}\DecValTok{10}\NormalTok{,}\DecValTok{10}\NormalTok{,}\DecValTok{10}\NormalTok{,}\DecValTok{10}\NormalTok{,}\DecValTok{11}\NormalTok{,}\DecValTok{11}\NormalTok{,}\DecValTok{11}\NormalTok{,}\DecValTok{11}\NormalTok{)}

\CommentTok{# 12 games per strategy }
\NormalTok{output12 <-}\StringTok{ }\KeywordTok{build_output}\NormalTok{(two_deck,}\DecValTok{12}\NormalTok{)}
\NormalTok{output12_avg <-}\StringTok{ }\KeywordTok{c}\NormalTok{(}\StringTok{"stand_16"}\NormalTok{=}\KeywordTok{mean}\NormalTok{(output12}\OperatorTok{$}\NormalTok{stand_}\DecValTok{16}\NormalTok{),}
                    \StringTok{"stand_17"}\NormalTok{=}\KeywordTok{mean}\NormalTok{(output12}\OperatorTok{$}\NormalTok{stand_}\DecValTok{17}\NormalTok{),}
                    \StringTok{"stand_18"}\NormalTok{=}\KeywordTok{mean}\NormalTok{(output12}\OperatorTok{$}\NormalTok{stand_}\DecValTok{18}\NormalTok{),}
                    \StringTok{"stand_19"}\NormalTok{=}\KeywordTok{mean}\NormalTok{(output12}\OperatorTok{$}\NormalTok{stand_}\DecValTok{19}\NormalTok{),}
                    \StringTok{"stand_20"}\NormalTok{=}\KeywordTok{mean}\NormalTok{(output12}\OperatorTok{$}\NormalTok{stand_}\DecValTok{20}\NormalTok{))}
\end{Highlighting}
\end{Shaded}

\begin{longtable}[]{@{}lr@{}}
\caption{Winnings by Strategy}\tabularnewline
\toprule
stand\_16 & 12.83\tabularnewline
stand\_17 & 10.25\tabularnewline
stand\_18 & 8.00\tabularnewline
stand\_19 & 10.17\tabularnewline
stand\_20 & 9.92\tabularnewline
\bottomrule
\end{longtable}

If you were to only look at the average winnings across 12 games, it
would appear that standing at 16 is the best option when the dealer
stands at 17. The below plot though illustrates just how much variation
there is in results. Simulations of 100, 500, 1000, and 1,500 rounds are
made below to improve decision-making.

\includegraphics{final_all_609_files/figure-latex/unnamed-chunk-8-1.pdf}

\paragraph{Increasing the Number of
Simulations}\label{increasing-the-number-of-simulations}

\begin{Shaded}
\begin{Highlighting}[]
\KeywordTok{set.seed}\NormalTok{(}\DecValTok{2345}\NormalTok{)}
\CommentTok{# 100 simulations}
\NormalTok{output100 <-}\StringTok{ }\KeywordTok{build_output}\NormalTok{(two_deck,}\DecValTok{100}\NormalTok{)}

\CommentTok{# 500 simulations}
\NormalTok{output500 <-}\StringTok{ }\KeywordTok{build_output}\NormalTok{(two_deck,}\DecValTok{500}\NormalTok{)}

\CommentTok{# 1000 simulations}
\NormalTok{output1000 <-}\StringTok{ }\KeywordTok{build_output}\NormalTok{(two_deck,}\DecValTok{1000}\NormalTok{)}

\CommentTok{# 1500 simulations}
\NormalTok{output1500 <-}\StringTok{ }\KeywordTok{build_output}\NormalTok{(two_deck,}\DecValTok{1500}\NormalTok{)}
\end{Highlighting}
\end{Shaded}

\includegraphics{final_all_609_files/figure-latex/unnamed-chunk-10-1.pdf}

\begin{longtable}[]{@{}rrrrrr@{}}
\caption{Average Winnings by Strategy and Simulation
Size}\tabularnewline
\toprule
Simulation\_Size & stand\_16 & stand\_17 & stand\_18 & stand\_19 &
stand\_20\tabularnewline
\midrule
\endfirsthead
\toprule
Simulation\_Size & stand\_16 & stand\_17 & stand\_18 & stand\_19 &
stand\_20\tabularnewline
\midrule
\endhead
100 & 10.15 & 10.96 & 13.26 & 11.63 & 9.23\tabularnewline
500 & 10.45 & 12.48 & 13.11 & 11.50 & 8.44\tabularnewline
1000 & 10.01 & 12.21 & 13.74 & 11.62 & 8.71\tabularnewline
1500 & 10.19 & 12.69 & 13.09 & 11.65 & 8.04\tabularnewline
\bottomrule
\end{longtable}

As the number of simulations increases, the results converge. It appears
to be that standing at 18 is the best option, as standing at 19 leaves
the player with fewer winnings. Similarly, standing at 16, 17, or 20
results in fewer winnings overall. Therefore if the dealer stands at 17,
the player should stand at 18.

\newpage

\section{Textbook Part III:}\label{textbook-part-iii}

\subsection{The problem}\label{the-problem}

The digestive processes of sheep can highlight the nutritionally value
in varied feeding schedules or varied food preparation. This is
especially important when raising sheep for commercial purposes.

\subsection{The digestive process}\label{the-digestive-process}

Sheep are a cud-chewing animal which means that unchewed food goes
through a series of storage stomachs called the rumen and the reticulum.
The process is illustrated below:

\begin{tikzpicture}[node distance = 2cm]
  \tikzstyle{process} = [rectangle, minimum width=3cm, minimum height=1cm, text centered, draw=black, fill=orange!30]
  \tikzstyle{arrow} = [thick,->,>=stealth]
  \node (r) [process, xshift=2cm] {Rumen};
  \node (a) [process, right of=r, xshift=2cm] {Abomasum};
  \node (d) [process, right of=a, xshift=2cm] {Duodemum};
  \node (f) [process, right of=d, xshift=2cm]  {Feces};
  \draw [arrow] (r) -- (a);
  \draw [arrow] (a) -- (d);
  \draw [dotted] (d) -- (f);
\end{tikzpicture}

\subsection{The experiment}\label{the-experiment}

The digestive process is most observable at the beginning and at the
end, we can observer and control what goes in and what comes out.

\subsection{The model}\label{the-model}

Suppose that at \(t = 0\) a sheep is fed an amount \(R\) of food which
goes immediately into its rumen. This food will pass gradually from the
rumen through the abomasum into the duodenum. At any later time \(t\) we
shall define:

\(r(t)\) = the amount of food still in the rumen; \(a(t)\) = the amount
in the abomasum; \(d(t)\) = the amount which by then has arrived in the
duodenum.

So \(r(0) = R\), \(a(0)\) = \(d(0)\) = 0, and, for all t\textgreater{}0,

~~~~~(1) \(r(t) + a(t) + d(t) = R.\)

\subsection{The assumptions}\label{the-assumptions}

Two assumptions are made:

\begin{enumerate}
\def\labelenumi{(\Alph{enumi})}
\tightlist
\item
  Food moves out of the rumen at a rate proportional to the amount of
  food in the rumen. Mathematically this says:
\end{enumerate}

~~~~~(2) \(r'(t) = -k_1r(t)\)

where \(k_1\) is a positive proportionality constant.

\begin{enumerate}
\def\labelenumi{(\Alph{enumi})}
\setcounter{enumi}{1}
\tightlist
\item
  Food moves out of the abomasum at a rate proportional to the amount of
  food in the abomasum. Since at the same time food is moving into the
  abomasum at the rate given by Equation (2), the assumption says
\end{enumerate}

~~~~~(3) \(a'(t) = k_1r(t)-k_2a(t)\)

where \(k_2\) is another positive proportionality constant.

\subsection{The solutions of the
equations}\label{the-solutions-of-the-equations}

\subsubsection{\texorpdfstring{Solving for
\(r(t)\)}{Solving for r(t)}}\label{solving-for-rt}

It is straightforward to solve Equation (2) for \(r(t)\).\\
We just divide through by \(r(t)\) and then integrate from 0 to t:

\[\int_0^t \frac{r'(t)}{r(t)} dt= - \int_0^t k_1dt\]\\
\[ln(\frac{r(t)}{R}) = -k_1t\],

since r(0) = R, and finally:

~~~~~(4) \(r(t) = Re^{-k_1t}\)

\subsubsection{\texorpdfstring{Solving for
\(a(t)\)}{Solving for a(t)}}\label{solving-for-at}

Finding \(a(t)\) is a bit more tricky. Applying Equations (4) to
Equation (3) we get:

~~~~~(5) \(a'(t) = k_1Re^{-k_1t}-k_2a(t)\)

Equation (5) probably looks quite different from any you have seen
before. Let us try to make a shrew guess what kind of solution it has.
It says that the derivative of \(a(t)\) is the sum of two terms,
\(k_1Re^{-k_1t}\) and \(-k_2a(t)\). With luck, this might remind us of
the product rule:

~~~~~(6) if \(a(t) = u(t) \bullet v(t)\) then
\(a'(t) = u(t) \bullet v'(t) + v(t) \bullet u'(t)\)

Can we pick \(u(t)\) and \(v(t)\) so the terms in Equation (6) match up
with the terms in Equation (5)? In other words, can we pick \(u(t)\) and
\(v(t)\) so that

~~~~~(7) \(u(t) \bullet v'(t) = k_1Re^{-k_1^t}\)

and

~~~~~(8) \(v(t) \bullet u'(t) = -k_2a(t)\)

Since \(a(t) = u(t) \bullet v(t)\), Equation (8) can rewritten
\(v(t) \bullet u'(t) = -k_2u(t)v(t)\), we are in business! The \(v(t)\)
factors cancel out, leaving us with

\[u'(t) = -k_2u(t)\]

which looks very much like Equation (2) and can be solved in the same
way.

\(\int_0^t \frac{u'(t)}{u(t)} dt = -\int_0^t k_2dt\)

Writing K = u(0):

\(ln (\frac{u(t)}{K}) = -k_2t\)\\
\(u(t) = Ke^{-k_2t}\).

Putting this into Equation (7) gives

\(Ke^{-k_2t}v'(t) = k_1Re^{-k_1t}\)\\
\(v'(t) = \frac{k_1R}{K}e(k_2-k_1)^t\).

If \(k_1 = k_2\) we feel confident you can complete this solution
yourself (Exercise 1).

\subsection{\texorpdfstring{Exercise 1. Find \(a(t)\) if
\(k_1 = k_2\).}{Exercise 1. Find a(t) if k\_1 = k\_2.}}\label{exercise-1.-find-at-if-k_1-k_2.}

\[a'(t) = u(t) \bullet v'(t) + v(t) \bullet u'(t)\]\\
Using the solution from equation 8 we get: \[u(t) = Ke^{-k_2t}\]\\
Using the solution from equation 7 we get:
\[v'(t) = \frac{k_1R}{K}e^{k_2 - k_1}t\]\\
Substituting the solutions for \(a(t)\) we get:
\[\frac{k_1R}{K}e^{k_2 - k_1}t * Ke^{-k_2t}\]\\
If k\_1 = k\_2 then \(e^{0} = 1\)\\
\[\frac{k_1R}{K}t * Ke^{-k_2t}\]\\
K factors out : \[k_1Rte^{-k_2t}\]

Solution : \[a(t) = k_1Rte^{-k_2t}\]

\begin{center}\rule{0.5\linewidth}{\linethickness}\end{center}

If \(k_1 \neq k_2\), then \(k_2 - k _1 \neq 0\) and so we can write

\[v(t) = \frac{k_1R}{K(k_2 - k_1)}e^(k_2 -k_1)^t +c\]

Where C is the constant of integration. Then

\[a(t) = u(t) \bullet v(t) = \frac{k_1R}{k_2-k_1}e^{-k_1t} + CKe^{-k_2t}\]

Using the fact that \(a(0) = 0\), we get

\[0 = \frac{k_1R}{k_2-k_1} + CK\] \[CK = -\frac{k_1R}{k_2-k_1}\] (9)
\(a(t) = \frac{k_1R}{k_2-k_1}(e^{-k_1t} - e^{-k_2t})\)

\subsection{Exercise 2.}\label{exercise-2.}

\subsubsection{\texorpdfstring{(a) Find the time t at which \(a(t)\) is
maximum.}{(a) Find the time t at which a(t) is maximum.}}\label{a-find-the-time-t-at-which-at-is-maximum.}

\[a(t) = \frac{k_1 R}{k_2 - k_1}(e^{-k_1 t}-e^{-k_2 t})\]\\
\[0 = \frac{k_1 R}{k_2 - k_1}(e^{-k_1 t}-e^{-k_2 t})\]\\
Divide both sides by \(\frac{k_1 r}{k_2 - k_1}\)\\
\[e^{-k_2 t}k2 - e^{-k_1 t}k1 = 0\]\\
Factor \(e^{-k_1 t}\), \(e^{-k_2 t}\):\\
\[-e^{-k_1 t - k_2 t}(e^{k_2 t}k_1 - e^{k_1 t}k2) = 0\]\\
Multiply by -1 to simplify and split into two equations:\\
\[e^{-k_1 t - k_2 t} = 0 ~and~e^{k_2 t}k_1 - e^{k_1 t}k2= 0\]\\
Since \(e^(z)\) can never be zero for any real number, no solution
exists for \[e^{-k_1 t - k_2 t} \neq 0\]\\
Divide both sides by \(e^{k2 t}\):\\
\[k_1 - e^{t (k_1 - k_2)}k2= 0\]\\
Subtract k1:\\
\[- e^{t (k_1 - k_2)}k2= -k_1\]\\
Divide both sides by -k2:\\
\[e^{t (k_1 - k_2)}= \frac{k_1}{k_2}\]\\
Take the log: \[t (k_1 - k_2)= \log{(\frac{k_1}{k_2})}\]\\
Solve for t: \[t = \frac{\log{(\frac{k_1}{k_2})}}{(k_1 - k_2)}\]

Alternate Solution:

\[t = \frac{lnk_1 - lnk_2}{k_1 - k_2}\]

\newpage

\subsubsection{\texorpdfstring{(b) Find the maximum value of
\(a(t)\).}{(b) Find the maximum value of a(t).}}\label{b-find-the-maximum-value-of-at.}

\[a(t) = \frac{k_1R}{k_2-k_1}(e^{-k_1t}-e^{-k_2t})\]

Substituting t we get :
\[\frac{k1 R}{k2 - k1}(e^{-k1 \frac{\log{(\frac{k1}{k2})}}{(k1 - k2)}}-e^{-k2 \frac{\log{(\frac{k1}{k2})}}{(k1 - k2)}})\]
Using the chain rule we identify that:
\[e^{-k1 \frac{\log{(\frac{k1}{k2})}}{(k1 - k2)}}=\frac{k_1}{k_2}^{-\frac{k_1}{k_1 - k_2}}\]
Therefore with substitution and applying the chain rule the maximum
value of \(a(t)\) is :

\[\frac{k_1R}{k_2-k_1}[(\frac{k_1}{k_2})^{-\frac{k_1}{k_1-k_2}}-(\frac{k_1}{k_2})^{-\frac{k_2}{k_1-k_2}}]\]

\subsection{Exercise 3.}\label{exercise-3.}

If \(k_1\) = 2 and \(k_2\) = 1, how much food must the abomasum be able
to hold if a meal of amount R is fed at time t = 0? If we substitute our
values we have:

\[\frac{2 R}{2 - 1}[(\frac{2}{1})^{-\frac{1}{2-1}}-(\frac{2}{1})^{-\frac{2}{2-1}}]\]
Simplify:

\[2[(\frac{2}{1})^{-1}-(\frac{2}{1})^{-2}]R\]
\[2[(\frac{1}{2})-(\frac{1}{4})]R\] \[2[(\frac{2}{4})-(\frac{1}{4})]R\]
\[2[\frac{1}{4}]R\] answer: \[\frac{1}{2}R\]

\newpage

\section{Comparison of the Model's Predictions with Experimental
Data}\label{comparison-of-the-models-predictions-with-experimental-data}

Equations (4) and (9) are purely theoretical and are based on
assumptions. To confirm their accuracy we will see if they agree with
experimental data. Since the data concern is fecal excretion as a
function of time, we must first translate our results into results on
fecal excretion.

\subsection{\texorpdfstring{Solving for
\(d(t)\)}{Solving for d(t)}}\label{solving-for-dt}

Starting with Equation (1), and using Equations (4) and (9) (still
assuming \(k_1 \neq k_2\)):

\[d(t) = R - r(t) - a(t)\]

\[= R - Re^{-k_1t} - \frac{k_1R}{k_2 - k_1} (e^{-k_1t}-e{-k_2t})\]

\[= R - \frac{R}{k_2-k_1}(k_2e^{-k_1t} - k_1e^{-k_2t})\]

\subsection{The Formula for the Amount of
Feces}\label{the-formula-for-the-amount-of-feces}

Recall that \(d(t)\) is the total amount of food which has entered the
duodenum by time t, including food already excreted. Since excretion is
not a continuous process, we cannot hope to represent it by an equation
involving derivatives. Instead, we simply note that all food arriving in
the duodenum is excreted after a certain time delay. Let us suppose that
the average time delay is T hours. In other words, the amount of feces
produced by any time t \textgreater{} T is, on the average, the amount
of food which had already entered the duodenum T hours earlier, at time
t - T. If f(t) denotes the amount of feces produced by time t, this
says:

\(f(t) = d(t-T) for all t > T\)

\begin{enumerate}
\def\labelenumi{(\arabic{enumi})}
\setcounter{enumi}{9}
\tightlist
\item
  \(f(t) = R - \frac{R}{k_2-k_1}(k_2e^{-k_1(t-T)}-k_1e^{-k_2(t-T)})\)
\end{enumerate}

for all t \textgreater{} T.

\subsection{Exercise 4.}\label{exercise-4.}

Remembering that r(t) is the total amount of food which has entered the
the rumen by time t. There is a continuous flow from the rumen to the
abomasum

\subsubsection{(a) Determine the values of t for which r'(t) is,
respectively, positive, negative, and
zero.}\label{a-determine-the-values-of-t-for-which-rt-is-respectively-positive-negative-and-zero.}

\[r(t) = Re^{-k_1t}\] Factor out constants:
\[=R(\frac{d}{dt}(e^{-k_1 t}))\] Using the chain rule :
\[= e^{-tk_1}(\frac{d}{dt}(-tk_1))R\] factor out constant:
\[=-k_1 (\frac{d}{dt}(t))e^{-k_1 t}R\] derivative of t is 1:
\[r'(t)= -e^{-tk_1}k_1R\] \(r'(t) = -e^{-t k_1}\) and therefore
\(r'(t) < 0\) for all \(t\).

\subsubsection{(b) Determine the values of t for which r''(t) is,
respectively, positive, negative, and
zero.}\label{b-determine-the-values-of-t-for-which-rt-is-respectively-positive-negative-and-zero.}

\[= \frac{d}{dt}(R(\frac{d}{dt}(e^{-tk_1})))\]\\
Using the chain rule:
\[= \frac{d}{dt}(e^{-tk_1}(\frac{d}{dt}(-tk_1))R)\]\\
factor out constants:
\[= \frac{d}{dt}(R -(\frac{d}{dt}(t))k_1 e^{-tk_1})\]\\
derivative of t is 1: \[= \frac{d}{dt}(R(e^{-tk_1}-k_1))\]\\
factor out constants: \[= R(\frac{d}{dt}(e^{-tk_1}))k_1\]\\
using the chain rule: \[= -Rk_1e^{-tk_1}(\frac{d}{dt}(-tk_1))\]\\
factor out constants and simplify
\[= -e^{-tk_1}R-(\frac{d}{dt}(t))k_1^2\]\\
derivative of t is 1: \[r''(t)= e^{-tk_1}Rk_1^2\]

\(r''(t) = e^{-t k_1}\) therefore for all real solutions :
\(r''(t) > 0\) for all \(t\)

\subsubsection{\texorpdfstring{(c) Determine
\(\lim_{t\to\infty}r(t)\)}{(c) Determine \textbackslash{}lim\_\{t\textbackslash{}to\textbackslash{}infty\}r(t)}}\label{c-determine-lim_ttoinftyrt}

\[r(t) = Re^{-k_1t}\] \[r(t) = Re^{-k_1 \infty}\] Since \(e^{-\infty}\)
= 0 our solution is: \(\lim_{t\to\infty}r(t) =0\)

\subsection{Exercise 5.}\label{exercise-5.}

Note that for exercise 5, 6, and 7
\(t_0 = \frac{ln k_1 - ln k_2}{k_1 - k_2}\)

Remembering that a(t) is the total amount of food which has entered the
the abomasum by time t from the rumen and exiting to the duodenum. There
is a continuous flow from the abomasum to the duodenum.

We have previously solved for a(t) and \(a'(t)\)

\[a(t) = \frac{k_1 R}{k_2-k_1}(e^{-k_1t} - e^{-k_2t})\]
\[a'(t) = k_1Re^{-k_1t}-k_2a(t)\]
\[= \frac{d}{dt}(\frac{(e^{-k_1 t}- e^{-k_2 t})k_1 R}{-k_1 + k_2})\]
Factor out constant:
\[=\frac{k_1 R (\frac{d}{dt}(e^{-k_1 t})-(e^{-k_2 t}))}{-k_1 + k_2}\]
Differentiate the sum term by term:
\[=\frac{\frac{d}{dt}(e^{-k_1 t})- \frac{d}{dt}(e^{-k_2 t})k_1 R}{-k_1 + k_2}\]
Use the chain rule:
\[\frac{k_1 R (-(\frac{d}{dt}(e^{-k_2t}))+\frac{\frac{d}{dt}-k_1t}{e^{k_1t}}}{-k_1 +k_2}\]
Factor out constants:
\[\frac{k_1 R (-(\frac{d}{dt}(e^{-k_2t}))+\frac{-k_1\frac{d}{dt}t}{e^{k_1t}}}{-k_1 +k_2}\]

The derivative of t is 1 and using the chain rule:
\[\frac{k_1 R(-\frac{k_1}{e^{k_1 t}}-\frac{\frac{d}{dt}-k_2t}{e^{k_1t}}}{-k_1 +k_2}\]
Factor out constants and the derivative of t is 1:
\[= \frac{k_1 ( -\frac{k_1}{e^{k_1t}}+\frac{k_2}{e^{k_2t}})R}{-k_1 + k_2}\]
\[a'(t) = \frac{k_1 ( -\frac{k_1}{e^{k_1t}}+\frac{k_2}{e^{k_2t}})R}{-k_1 + k_2}\]

\subsubsection{\texorpdfstring{(a) Determine the values of t for which
\(a'(t)\) is, respectively, positive, negative, and
zero.}{(a) Determine the values of t for which a'(t) is, respectively, positive, negative, and zero.}}\label{a-determine-the-values-of-t-for-which-at-is-respectively-positive-negative-and-zero.}

\[a'(t) = \frac{k_1 ( -\frac{k_1}{e^{k_1t}}+\frac{k_2}{e^{k_2t}})R}{-k_1 + k_2}\]
\[a'(t) = \frac{k_1 ( -\frac{k_1}{e^{k_1t}}+\frac{k_2}{e^{k_2t}})R}{-k_1 + k_2}\]
\[= \frac{k_1 ( -\frac{k_1}{e^{k_1 t_1}}+\frac{k_2}{e^{k_2 t_1}})R}{-k_1 + k_2}\]

Therefore assuming that:

\(k_1 > 0\) and \(k_2 > k_1\) and
\(t_0 = \frac{log(\frac{k_1}{k_2})}{k_1 - k_2}\); \(a'(t)\) is positive
or \(a'(t) > 0\) for all \(t < t_0\);

\[= \frac{k_1 ( -\frac{k_1}{e^{k_1*0}}+\frac{k_2}{e^{k_2*0}})R}{-k_1 + k_2}\]
\(t_0 = \frac{ln k_1 - ln k_2}{k_1 - lnk_2}\)

\[= \frac{k_1 ( -\frac{k_1}{e^{k_1*\frac{ln k_1 - ln k_2}{k_1 - lnk_2}}}+\frac{k_2}{e^{k_2*\frac{ln k_1 - ln k_2}{k_1 - lnk_2}}})R}{-k_1 + k_2}\]

\[\frac{k_1 R (k_1 \frac{k_1}{k_2}^{\frac{k_1}{log(k_2)-k_1}} - k_2 \frac{k_2}{k_1}^{\frac{k_2}{k_1 - log(k_2)}})}{k_1 - k_2}\]

Therefore : \(a'(t_0) = 0\);

We know from our previous exercise we can conclude that \(a'(t)\) is
negative or \(a'(t) < 0\) for \(t > t_0\).

\subsubsection{(b) Determine the values of t for which a''(t) is,
respectively, positive, negative, and
zero.}\label{b-determine-the-values-of-t-for-which-at-is-respectively-positive-negative-and-zero.}

\[a(t) = \frac{k_1 R}{k_2-k_1}(e^{-k_1t} - e^{-k_2t})\] Differentiate
the sum, factor constants, and use the chain rule :

\[\frac{d}{dt}(\frac{k_1 R(e^{-k_1t}\frac{d}{dt}(-k_1t)-\frac{d}{dt}(e^{-k_2t}))}{-k_1 + k_2})\]
Factor out constants, derivative of t is 1, and use the chain rule :
\[\frac{d}{dt}(\frac{k_1R(e^{-k_1 t}(-k_1)- e^{-k_2t}\frac{d}{dt}(-k_2t))}{-k_1 + k_2})\]
Factor out constants, derivate of t is 1, and differentiate the sum
term:
\[= k_2 \frac{d}{dt}(e^{-k_2 t})-k_1\frac{d}{dt}(e^{-k_1t})\frac{k_1R}{-k_1 + k_2}\]
Using the chain rule, factor out constants, derivative of t is 1:
\[=\frac{k_1 R (k_2 \frac{d}{dt}(e^{-k_2t}))+e^{-k_1t}k_1^2)}{-k_1 + k_2}\]
Using the chain rule, factor out constants, and the derivative of t is
1: \[a''(t) = \frac{k_1(e^{-k_1 t}k_1^2-e^{-k_2 t}k_2^2)R}{-k_1+k_2}\]
Note that \(t_0 = \frac{ln k_1 - ln k_2}{k_1 - k_2}\)

\[\frac{k_1(e^{-k_1 \frac{log(k1) - log(k2)}{k1 - k2}}k_1^2-e^{-k_2 \frac{log(k1) - log(k2)}{k1 - k2}}k_2^2)R}{-k_1+k_2}\]
\[\frac{k_1 R(k_1^2(\frac{k_1}{k_2}^{\frac{k_1}{k_1 - k_2}} - k_2^2(\frac{k_1}{k_2})^{-\frac{k_2}{k_1 - k_2}}))}{k_2 - k_1}\]
\[-k_1 k_2 R (\frac{k_1}{k_2}^{-\frac{k_1}{k_1 - k_2}})\]

Therefore we can see that \(a''(t) < 0\) for all \(t < 2t_0\);
\[\frac{k_1(e^{-k_1 2 \frac{log(k1) - log(k2)}{k1 - k2}}k_1^2-e^{-k_2 2\frac{log(k1) -log(k2)}{k1 - k2}}k_2^2)R}{-k_1+k_2}\]
Assuming \(k_1\), \(k_2\), and \(R\) are real numbers then
\(a''(2t_0) =0\);

As we saw from the previous exercise that \(a''(t) > 0\) for
\(t > 2t_0\).

\subsubsection{\texorpdfstring{(c) Determine
\(\lim_{t\to\infty}a(t)\)}{(c) Determine \textbackslash{}lim\_\{t\textbackslash{}to\textbackslash{}infty\}a(t)}}\label{c-determine-lim_ttoinftyat}

\[a(t) = \frac{k_1 R}{k_2-k_1}(e^{-k_1t} - e^{-k_2t})\]
\[a(t) = \frac{k_1 R}{k_2-k_1}(e^{-k_1 \infty} - e^{-k_2 \infty})\]
Again, as we saw previously \(e^{-\infty}\) = 0 so our solution :
\(\lim_{t\to\infty}a(t) =0\)

\subsection{Exercise 6.}\label{exercise-6.}

\[d(t) = R - r(t) - a(t)\]
\[d(t) = R - \frac{R}{k_2 - k_1}(k_2 e ^{-k_1 t} - k_1 e^{-k_2 t})\]

\subsubsection{(a) Determine the values of t for which d'(t) is,
respectively, positive, negative, and
zero.}\label{a-determine-the-values-of-t-for-which-dt-is-respectively-positive-negative-and-zero.}

\[d(t) = R - \frac{R}{k_2 - k_1}(k_2 e ^{-k_1 t} - k_1 e^{-k_2 t})\]
\[- \frac{R (\frac{d}{dt} (-e^{-k_2 t} k1 + e^{-k1t}k2))}{-k_1 + k_2} + \frac{d}{dt} (R)\]\\
\[\frac{d}{dt}(R) - k_2 \frac{d}{dt}(e^{-k_1 t})-k1\frac{d}{dt}(e^{-k_2 t})\frac{R}{-k_1 + k_2}\]
Using the chain rule and derivative of R is zero:
\[- \frac{R(-k_1 \frac{d}{dt}(e^{-k_2 t})) + e^{-k_1 t}\frac{d}{dt}(-k_1 t)k_2)}{-k_1 + k_2}\]
Factor out constants and derivative of t is 1:
\[- \frac{R(-k_1 \frac{d}{dt}(e^{-k_2 t})) + 1e^{-k_1 t}k_2)}{-k_1 + k_2}\]
Using chain rule:
\[- \frac{R(-e^{-k_1 t}k_1 k_2 - e^{-k_2 t} \frac{d}{dt} (-k_2 t)k_1}{-k_1 + k_2}\]
Factor out:
\[- \frac{R(-e^{-k_1 t}k_1 k_2 - -k_2 \frac{d}{dt}(t)e^{-k_2 t}k_1)}{-k_1 + k_2}\]
Simplify and derivative of t is 1:
\[d'(t) = -\frac{(-e^{-k_1 t}k_1 k_2 + e^{-k_2 t}k_1 k_2)R}{-k_1 + k_2}\]
Note \(t_0 = \frac{ln k_1 - ln k_2}{k_1 - k_2}\)

\[-\frac{(-e^{-k_1 \frac{ln k_1 - ln k_2}{k_1 - k_2}}k_1 k_2 + e^{-k_2 \frac{ln k_1 - ln k_2}{k_1 - k_2}}k_1 k_2)R}{-k_1 + k_2}\]
\[\frac{k_1 k_2 R (\frac{k_2}{k_1}^{\frac{k_2}{k_1 - k_2}}-\frac{k_2}{k_1}^{\frac{k_1}{k_1 - k_2}})}{k_1 - k_2}\]
\[k_2R(\frac{k_2}{k_1})^{\frac{k_2}{k_1 - k_2}}\]

Therefore we can conclude that \(d'(t)\) is positive or \(d'(t) > 0\)
for all t (must have \(k_1 > k_2\)).

\subsubsection{(b) Determine the values of t for which d''(t) is,
respectively, positive, negative, and
zero.}\label{b-determine-the-values-of-t-for-which-dt-is-respectively-positive-negative-and-zero.}

\[d'(t) = -\frac{(-e^{-k_1 t}k_1 k_2 + e^{-k_2 t}k_1 k_2)R}{-k_1 + k_2}\]\\
\[= -\frac{R(k_1 k_2 (\frac{d}{dt}(e^{-k_2 t})) - -k_1 \frac{d}{dt}(t)e^{-k_1 t}k_1 k_2 )}{-k_1 + k_2}\]
Simplify and the derivative of t is 1:
\[= -\frac{R(k_1 k_2 (\frac{d}{dt}(e^{-k_2 t})) - e^{-k_1 t}k_1^2 k_2 )}{-k_1 + k_2}\]\\
Use the chain rule:
\[= -\frac{R(e^{-k_1 t}k_1^2 k_2 + e^{-k_2 t} \frac{d}{dt}(t)k_1 k_2 }{-k_1 + k_2}\]\\
Factor out constants:
\[= -\frac{R(e^{-k_1 t}k_1^2 k_2 + - k_2 \frac{d}{dt}(t) e^{-k_2 t}k_1 k_2} {-k_1 + k_2}\]\\
Derivative of t is 1 and simplify:
\[= - \frac{R (e^{-k_1 t}k_1^2 k_2) - e^{k_2 t} k_1 k_2^2)}{-k_1 + k_2}\]\\
\[d''(t) = \frac{k_1 k_2 R e^{t(-k_1-k_2)}(k_1 e^{k_2 t}-k_2 e^{k_1 t})}{k_1 - k_2}\]

\[(k_1 - k_2)(k_1 e ^{k_2 t}-k_2 e^{k_1 t}) > 0\]

Therefore \(d''(t)\) is positive or \(d''(t) > 0\) for all \(t < t_0\);

Note that \(t_0 = \frac{ln k_1 - ln k_2}{k_1 - k_2}\)

\[\frac{k_1 k_2 R e^{t(-k_1-k_2)}(k_1 e^{k_2 \frac{ln k_1 - ln k_2}{k_1 - k_2}}-k_2 e^{k_1 \frac{ln k_1 - ln k_2}{k_1 - k_2}})}{k_1 - k_2}\]

\[\frac{k_1 k_2 R(k_1 (\frac{k_1}{k_2})^{\frac{k_2}{k_1 - k_2}}-k_2(\frac{k_1}{k_2}^{\frac{k_1}{k_1 - k_2}}))e^{t(-k_1 - k_2)}}{k_1 - k_2}\]

\[\frac{k_1^2 k_2 R (\frac{k_1}{k_2})^{\frac{k_1}{k_1 - k_2}}e^{\frac{(-k_1 - k_2)(log(k_1) - log(k_2))}{k_1 - k_2}}}{k_1 - k_2} - \frac{k_1 k_2^2 R (\frac{k_1}{k_2})^{\frac{k_1}{k_1 - k_2}}e^{\frac{(-k_1 - k_2)(log(k_1) - log(k_2))}{k_1 - k_2}}}{k_1 - k_2}\]

Therefore \(d''(t)\) or \(d''(t_0) =0\);

If the first part of the exercise is indeed true then \(d''(t)\) is
negative or \(d''(t) < 0\) for \(t > t_0\).

\subsubsection{\texorpdfstring{(c) Determine
\(\lim_{t\to\infty}d(t)\)}{(c) Determine \textbackslash{}lim\_\{t\textbackslash{}to\textbackslash{}infty\}d(t)}}\label{c-determine-lim_ttoinftydt}

\[d(t) = R - \frac{R}{k_2 - k_1}(k_2 e ^{-k_1 t} - k_1 e^{-k_2 t})\]
\[d(t) = R - \frac{R}{k_2 - k_1}(k_2 e ^{-k_1 \infty} - k_1 e^{-k_2 \infty})\]
\[d(t) = R - \frac{R}{k_2 - k_1}(k_2 e ^{-k_1 \infty} - k_1 e^{-k_2 \infty})\]
Since \(e^{-\infty}\) = 0 :

\[d(t) = R - \frac{R}{k_2 - k_1}(k_2*0 - k_1*0)\] Therefore, our
solution is : \(\lim_{t\to\infty}d(t) =R\)

\subsection{Exercise 7.}\label{exercise-7.}

For the function f(t) follow the instructions of Exercise 4.

\[f(t) = R - \frac{R}{k_2 - k_1}(k_2 e^{-k_1(t-T)}-k_1 e^{-k_2(t-T)})\]

\subsubsection{(a) Determine the values of t for which f'(t) is,
respectively, positive, negative, and
zero.}\label{a-determine-the-values-of-t-for-which-ft-is-respectively-positive-negative-and-zero.}

\[=- \frac{R \frac{d}{dt}(-e^{-k_2 (t-T)}k_1 + e^{-k_1(t - T)}k_2))}{-k_1 + k_2} + \frac{d}{dt}(R)\]

Factor out constants:
\[=\frac{d}{dt}(R) - k_2(\frac{d}{dt}(e^{-k_1(t - T)})) - k_1 (\frac{d}{dt}(e^{-k_2(t-T)})) \frac{R}{-k_1 + k_2}\]
Using the chain rule:
\[=\frac{d}{dt}(R)-\frac{R(-k_1)(\frac{d}{dt}(e^{-k_2(t-T)})) + e^{-k_1(t-T)}\frac{d}{dt}(-k_1(t-T)))k_2)}{-k_1+ k_2}\]
Factor out constants:
\[=\frac{d}{dt}(R) - \frac{R(-k_1(\frac{d}{dt}(e^{-k_2(t-T)})))+-k_1(\frac{d}{d_t}(t-T))e^(-k_1(t-T))k_2)}{-k_1 + k_2}\]
Differentiate the sum term:
\[=\frac{d}{dt}(R) - \frac{R(-k_1(\frac{d}{dt}(e^{-k_2(t-T)})))+ \frac{d}{dt}(t) + \frac{d}{dt}(-T)e^{-k1(t-T)k_1 k_2}}{-k_1 + k_2}\]
Derivative of t is 1 and R is 0:
\[=- \frac{R(-k_1(\frac{d}{dt}(e^{-k_2(t-T)})))+ \frac{d}{dt}(-T)e^{-k1(t-T)k_1 k_2}}{-k_1 + k_2}\]
Using the chain rule:
\[=-\frac{R(-e^{-k_1 (t-T)}k_1 k_2(\frac{d}{dt}(-T)))- e^{-k_2(t-T)}(\frac{d}{dt}(-k_2(t-T)))k_1}{-k_1 + k_2}\]
Factor out constants:
\[=-\frac{R(-e^{-k_1 (t-T)}k_1 k_2(\frac{d}{dt}(-T)))- -k_2(\frac{d}{dt}(t-T))e^{-k_2 (t-T)}k_1}{-k_1 + k_2}\]
Differentiate the sum term:
\[=-\frac{R(-e^{-k_1(t-T)}k_1 k_2(\frac{d}{dt}(-T)) + \frac{d}{dt}(t)+\frac{d}{dt}(-T)e^{-k_2(t-T)}k_1 k_2)}{-k_1 + k_2}\]
The derivative of t is 1 and -T is zero
\[=-\frac{R(-e^{-k_1(t-T)}k_1 k_2 + e^{-k_2(t-T)}k_1 k_2)}{-k_1 + k_2}\]
\[f'(t)=-\frac{-e^{-k_1 (t -T)}k_1 k_2 + e^{-k_2(t-T)}k_1 k_2)R}{-k_1 + k_2}\]

note \(t_0 = \frac{ln k_1 - ln k_2}{k_1 - k_2}\).

\[(k_1 - k_2)(e^{k_1 (T-t)}-e^{k_2(T-t)})>0\] Therefore \(f'(t)\) is
positive or \(f'(t) > 0\) for all \(t > T\).

\subsubsection{(b) Determine the values of t for which f''(t) is,
respectively, positive, negative, and
zero.}\label{b-determine-the-values-of-t-for-which-ft-is-respectively-positive-negative-and-zero.}

\[=\frac{d}{dt}(\frac{d}{dt}(R - \frac{(-e^{-k_2(t-T)}k_1) + e^{-k_1(t-T)}k_2)R}{-k_1+k_2}))\]
Differentiate and factor out:
\[=\frac{d}{dt}(\frac{d}{dt}(R) - \frac{R \frac{d}{dt}(-e^{-k_2(t-T)}k_1 + e^{-k_1(t-T)}k_2)}{-k_1+k_2}))\]
Factor out:
\[=\frac{d}{dt}(-\frac{R}{-k_1 + k_2}k_2\frac{d}{dt}(e^{-k_1(t-T)})-k_1\frac{d}{dt}(e^{-k_2(t-T)})+\frac{d}{dt}(R))\]
Using the chain rule:
\[=\frac{d}{dt}(-\frac{R(e^{-k_1(t-T)}\frac{d}{dt}(-k_1(t-T))k_2 - k_1(\frac{d}{dt}(e^{-k_2(t-T)}))}{-k_1 +k_2}+\frac{d}{dt}(R))\]
Factor out:
\[= \frac{d}{dt}(-\frac{R(k_2-k_1 \frac{d}{dt}(t-T)e^{-k_1(t-T)}-k_1( \frac{d}{dt}(e^{-k_2(t-T)})))}{-k_1+k_2} + \frac{d}{dt}(R))\]
Differentiate the sum
\[=\frac{d}{dt}(\frac{-R(k_2(e^{-k_1(t-T)})-k_1\frac{d}{dt}(t)+\frac{d}{dt}(-T)-k_1(\frac{d}{dt}(e^{-k_2(t-T)})))}{-k_1+k_2}+\frac{d}{dt}(R))\]
The derivative of t is 1 and -T is zero:
\[=\frac{d}{dt}(\frac{-R(k_2(e^{-k_1(t-T)})-k_1(1+0)-k_1(\frac{d}{dt}(e^{-k_2(t-T)})))}{-k_1+k_2}+\frac{d}{dt}(R))\]
Use the chain rule:
\[=\frac{d}{dt}(\frac{-R(k_2(e^{-k_1(t-T)})-e^{-k_2(t-T)}\frac{d}{dt}(-k_2(t-T))k_1 }{-k_1+k_2}+\frac{d}{dt}(R))\]
Factor out:
\[=\frac{d}{dt}(\frac{-R(k_2(e^{-k_1(t-T)})-k_1-k_2\frac{d}{dt}(t-T)e^{-k_2(t-T)})  }{-k_1+k_2}+\frac{d}{dt}(R))\]
Differentiate the sum:
\[=\frac{d}{dt}(\frac{-R(k_2(e^{-k_1(t-T)})-k_1(e^{-k_2(t-T)}-k_2\frac{d}{dt}(t)+\frac{d}{dt}(-T)}{-k_1+k_2}+\frac{d}{dt}(R))\]
The derivative of t is 1 and, -T and R is zero:
\[=\frac{d}{dt}(\frac{-R(k_2(e^{-k_1(t-T)})-k_1(e^{-k_2(t-T)}-k_2(1+0)}{-k_1+k_2}+0)\]
Factor out:
\[=-\frac{R\frac{d}{dt}(-e^{-k_1(t-T)}k_1 k_2 + e^{-k_2(t-T)}k_1 k_2))}{-k_1 + k_2}\]
Use the chain rule:
\[=-\frac{R(k_1 k_2(\frac{d}{dt}(e^{-k_2(t-T)}))-e^{-k_1(t-T)}\frac{d}{dt}(-k_1(t-T))k_1 k_2)}{-k_1 k_2}\]
Factor out:
\[=-\frac{R(k_1 k_2 \frac{d}{dt}(e^{-k_2(t-T)}))--k_1\frac{d}{dt}(t-T)e^{-k_1(t-T)}k_1 k_2}{-k_1 k_2}\]
Simply
\[=-\frac{R(k_1 k_2 \frac{d}{dt}(e^{-k_2(t-T)}))+\frac{d}{dt}(t)+\frac{d}{dt}(-T)e^{-k_1(t-T)}k_1^2 k_2}{-k_1 k_2}\]
The derivative of t is 1:
\[=-\frac{R(k_1 k_2 \frac{d}{dt}(e^{-k_2(t-T)}))+(1+\frac{d}{dt}(-T))e^{-k_1(t-T)}k_1^2 k_2}{-k_1 k_2}\]
Using the chain rule:
\[=-\frac{R(e^{-k_1(t-T)}k_1^2k_2(1+\frac{d}{dt}(t-T))+-k_2(\frac{d}{dt}(t-T)e^{-k_2(t-T)}k_1 k_2)}{-k_1 + k_2}\]
Factor and differentiate the sum term by term:
\[=-\frac{R(e^{-k_1(t-T)k_1^2k_2}(1+\frac{d}{dt}(-T))- \frac{d}{dt}(t)+\frac{d}{dt}(-T)e^{-k_2(t-T)}k_1 k_2^2)}{-k_1 + k_2}\]
Derivative of t is 1 and -T is zero:
\[=-\frac{R(e^{-k_1(t-T)k_1^2k_2}(1+0)-e^{-k_2(t-T)}k_1 k_2^2)(1+0)}{-k_1 + k_2}\]
Answer:
\[f''(t) =-\frac{(e^{-k_1 (t-T)}k_1^2 k_2 -e^{-k_2(t-T)}k_1 k_2^2)R}{-k_1 + k_2}\]

\[\frac{-R e ^{k_1 T - k_1 t}k_1^2 k_2 - R e^{k_2T -k_2t}k_1 k_2^2}{k_2 - k_1}\]
\[\frac{ e^{k_2 t - k_2 t}k_1 k_2^2 R}{k_2 - k_1} - \frac{ e^{k_1 t - k_1 t}k_1^2 k_2 R}{k_2 - k_1}\]
\[0 < (k_1 - k_2)(k_1 e^{k_1(T-t)}- k_2 e^{k_2(T-t)})\]

Therefore \(f''(t)\) is positive or \(f''(t) > 0\) for all
\(T < t < T + t_0\);

Note that \(t_0 = \frac{ln k_1 - ln k_2}{k_1 - k_2}\)

\[0 = \frac{k_1 e^{k1(\frac{log(\frac{k_2}{k_1})}{k_1 - k_2}+T)} + k_2 e^{k2(\frac{log(\frac{k_2}{k_1})}{k_1 - k_2}+T)} }{k_1 - k_2}\]
\[T = \frac{-log(\frac{k_1}{k_2}+log(k_1)- log(k_2)}{k_1 - k_2}\]
Therefore \(f''(t)\) is zero at \(f''(T + t_0) =0\);

Having found where \(f''(t)\) is positive we can then conclude that
\(f''(t)\) is negative or \(f''(t) < 0\) for \(t > T + t_0\).

\subsubsection{\texorpdfstring{(c) Determine
\(\lim_{t\to\infty}f(t)\)}{(c) Determine \textbackslash{}lim\_\{t\textbackslash{}to\textbackslash{}infty\}f(t)}}\label{c-determine-lim_ttoinftyft}

\[f(t) = R - \frac{R}{k_2 - k_1}(k_2 e^{-k_1(t-T)}-k_1 e^{-k_2(t-T)})\]
\[f(t) = R - \frac{R}{k_2 - k_1}(k_2 e^{-k_1(\infty-T)}-k_1 e^{-k_2(\infty-T)})\]
Since \(e^{-\infty}\) = 0:

\(f(t) = R\) so our solution is answer: \(\lim_{t\to\infty}f(t) = R\)

\subsection{Exercise 8.}\label{exercise-8.}

Using all the information in Exercise 4, 5, 6, and 7 to sketch the
graphs of r, a, d, and f. You should not have to compute any points
except r(0), a(0), d(0), and f(T), which you already know.

\includegraphics{https://raw.githubusercontent.com/dbouquin/DATA_609_final/master/20170507_152919.jpg}
\newpage

\subsection{Exercise 9.}\label{exercise-9.}

Assuming that increased stay in the digestive system implies improved
digestion determine from Figure 4 which method of food preparation is
more desirable. Why?

\begin{figure}
\centering
\includegraphics{https://raw.githubusercontent.com/dbouquin/DATA_609_final/master/Figure4.PNG}
\caption{}
\end{figure}

Assuming that we are looking for the feed type that stays in the sheep's
stomach the longest, then we would say that the medium-ground grass
would be the most desirable. We make this assumption as it stands that
the sheep requires less feed for the same results.


\end{document}
